% !TeX encoding = UTF-8

% 载入 SJTUThesis 模版
\documentclass[type=bachelor,lang=en, oneside]{sjtuthesis}
% 选项
%   type=[doctor|master|bachelor],     % 可选(默认:master),论文类型
%   zihao=[-4|5],                      % 可选(默认:-4),正文字号大小
%   lang=[zh|en|de|ja],                % 可选(默认:zh),论文的主要语言
%   review,                            % 可选(默认:关闭),盲审模式
%   [twoside|oneside],                 % 可选(默认:twoside),双页或单页边距模式
%   [openright|openany],               % 可选(默认:openright),奇数页或任意页开始新章
%   math-style=[ISO|TeX],              % 可选 (默认:ISO),数学符号样式
\usepackage[hidelinks]{hyperref}
\usepackage{etoolbox}

\usepackage{tikz-cd}  % 使用 TikZ 绘图



% 论文基本配置,加载宏包等全局配置
\input{setup}

\begin{document}
\setlength{\baselineskip}{20pt}

%TC:ignore

% 标题页
\maketitle

% 原创性声明及使用授权书
\copyrightpage
% 插入外置原创性声明及使用授权书
% 此时必须在导言区使用 \usepackage{pdfpages}
% \copyrightpage[scans/sample-copyright.pdf]

% 前置部分
\frontmatter

\clearpage
{
\ExplSyntaxOn
\bool_if:NTF \g__sjtu_twoside_bool
{
    \fancyhead [ LE ]     {  }
    \fancyhead [ RO ]     {  }
}
{
    \fancyhead [ R ] { }
}
\ExplSyntaxOff

{
\ctexset{chapter={afterskip=26bp}}
% 摘要
% !TEX root = ../main.tex

\begin{abstract}[zh]
\addcontentsline{toc}{chapter}{摘 \quad 要}
本文是建立在$(\infty,1)$-范畴语言中的对母体同伦理论的概述。

对于一个Noetherian概形$S$,存在一个不稳定的母体同伦范畴$\mathcal{H}(S)$,它可以视为预层范畴$Sm/S$的反射局部化。$\mathcal{H}(S)$ 中存在两种圆,这诱导了双分次的母体稳定同伦范畴$\mathcal{SH}_{s,t}(S)$和母体稳定同伦范畴$\mathcal{SH}(S)$。存在一个母体版本的Brown表示定理,它将(双分次)母体谱和(双分次)母体上同调理论联系起来。母体稳定同伦范畴$\mathcal{H}(S)$具有一个泛性质使每个上同调理论都通过它。此外,$\mathcal{SH}(S)$上存在一个6-函子理论。
\end{abstract}

{
\newfontface{\arial}{Arial}[Scale=0.94]
\ctexset{chapter/format+={\arial}}
\begin{abstract}[en]
\addcontentsline{toc}{chapter}{ABSTRACT}

This paper is an overview of motivic homotopy theory in the language of $(\infty,1)$-categories.

  For a Noethrian scheme $S$, there exists an unsabtle motivic homotopy category $\mathcal{H}(S)$  which can be considered as a reflective localization of the presheaf category $Sm/S$.  $\mathcal{H}(S)$ admits two types of circles, which induce bigraded motivic stable homotopy category $\mathcal{SH}_{s,t}(S)$ and motivic stable homotopy category $\mathcal{SH}(S)$. There is a motivic version Brown representibility theorem, which connects (bigraded) motivic spectra and (bigraded) motivic cohomology theory. The motivic stable homotopy category $\mathcal{SH}(S)$ has an universal property that every cohomology theory passes through it. Furthermore,  $\mathcal{SH}(S)$ admits a 6-functor formalism on it.

\end{abstract}
}
}

{
\ctexset{chapter={afterskip=26bp}}
{
\ctexset{chapter={afterskip=26bp}}
\renewcommand{\cftchapfont}{\zihao{4}\bfseries}
\renewcommand{\cftsecfont}{\zihao{-4}}
\renewcommand{\cftsubsecfont}{\zihao{5}}

% 目录
\tableofcontents
}
}
\clearpage
}

% % 插图索引
% \listoffigures*
% % 表格索引
% \listoftables*
% % 算法索引
% \listofalgorithms*
% % 符号对照表
% \input{contents/nomenclature}

%TC:endignore

% 主体部分
\mainmatter

% 正文内容
{
\ctexset{chapter={afterskip=26bp}}
% !TEX root = ../main.tex

\chapter{Introduction}
\section{Foreword}

The concept of ``motive" was introduced by Grothendieck who noticed a series of similarities in Weil cohomology theories. In his idea, he thought there should be an universal cohomology theory which every cohomolgy theory taking values in a $\mathbb{Q}$-category should pass through it.

This idea soon became so important in the research about Weil conjecture that people belive once the ``motive" is discovered, then the problem can be done.

Thanks to the effort of many mathematicians, we succeeded in constructing ``pure motives" for smooth projective varieties as well as ``mixed motives" for a more general cases.

One important observation in algebraic topology is that when we talk about cohomology theories, what really matters is the derived category. It was Voevodksy who introduced the ``motivic homotopy category" and prove the universal property of it, which is what we are interested in this paper.

\section{The main content of this paper}
In short, motivic homotopy theory concerns how to construct a algebraic topology theory on algebraic geometry. To give the readers an intuition, we include a correponding diagram below.

\begin{table}[ht]
\centering
\caption{Comparison between classic algebraic topology world and motivic world}
\begin{tabular}{|>{\centering\arraybackslash}m{4cm}|>{\centering\arraybackslash}m{4cm}|}
\hline
\textbf{Classic World} & \textbf{Motivic World} \\
\hline
$\mathscr{S}$ (Spaces) & $\mathcal{H}(S)$ (Motivic homotopy category) \\
\hline
$I$& $\mathbb{A}^1$ \\
\hline
$I$-homotopy equivalence & $\mathbb{A}^1$-homotopy equivalence \\
\hline
$S^1$& Tate sphere \\
\hline
$\textbf{Sptr}$ (Spectra) & $\mathcal{SH}(S)$ ( Stable motivic homotopy category) \\
\hline
Cohomology & Motivic cohomology \\
\hline
\end{tabular}
\end{table}

In Chapter 2, we will firstly recommend some preliminaries that are important to understand the topic we'll discuss. It mainly contains the language of $\infty-$categories, stable $\infty-$ categories and the theory of six functors.

In Chapter 3, we will consturct the unstable motivic homotopy category $\mathcal{H}(S)$ and explain the reasonability in the constuction. 

In Chapter 4, we will construct the stable motivic homotopy category $\mathcal{SH}(S)$, which is a slightly differnt from the standard stabilization process. The existence of two circles in $\mathcal{H}(S)$ give rise to bigraded cohomology theories in motivic world. We will also see those important properties(universal properties, six functor formalism) and some spectra which have essential applications in other fields.

\section{The significance of this article}
Since motivic homotopy theory is a topic developed in mid 1990s and the language where it is based has changed with the development of $\infty-$categories, there has been a lack of a very detailed introduction of this topic in modern language. This is what this paper is in intended for.


% !TEX root = ../main.tex

\chapter{Preliminaries}

In this chapter, we will recommend the prelimiaries about $\infty-$categories, stable $\infty-$categories and six functor theories. Readers who are already familiar with them can skip this chapter and refer to \ref{chap:symbol} for notations.

\section{Foundations of $\infty$-categories}
The motivation of $\infty$-categories is to consturct a language where we can talk about higher morphisms. A naive approach towards this is enriched category. However, the coherence data soon become too complicated to describe as the level increases. So we will construct some models to describe $\infty$-categories.

It should be noted what we call $\infty$- categories here are in fact $(\infty,1)$-categories, which means all n-morphisms ($n \geq 2$) are invertible.

Readers are assumed to have been familiar with the knowledge of simplicial sets and model categories.

\subsection{Models of $\infty$-categories}
When we say a ``model" for $\infty$- categories, we mean a model category, whose objects(up to weak equivalence) are $\infty$- categories and the weak equivalence between two objects represents equivalence between two $\infty$- categories. These models are ``equivalent", namely Quillen equivalent.

In short, 


    \textbf{The theory of $(\infty,1)$-categories lies in the homotopy category of a model category which ``models" the $\infty$-categories.} \label{homocat} \hfill (2.1)



 Although for practical reasons we have to construct those definitions and propositions in a specific category, they should be invariant under weak equivalences.

In this subsection, we will introduce two models of $\infty$-categories, which are quasi-categories and simplicially enriched categories.

\begin{definition}(Boardman-Vogt, 1973)
    A simplical set X is called a quasi-category if it satisfies the right lifting property aganist inner horn inclusions:
\[
\begin{tikzcd}
\Lambda^n_k \arrow[r] \arrow[d, hook] & X \arrow[d] \\
\Delta^n \arrow[r] \arrow[ur, dashed, "\exists h" description] & \ast
\end{tikzcd}
\qquad (0 < k < n)
\] 
\end{definition}

Take $\Lambda^n_k= \Lambda^2_1$, then the lifting property provides a lifting $h:\Delta^2\to X$. If we denote $f$ as the image of $\{0\to1\}$ in $X$, $g$ as the image of $\{1\to2\}$ in $X$, then we can view $h$ as the ``homotopy" witnessing the composition of $f$ and $g$.

A question arising here is that the ``homotopy" $h$ witnessing the composition is not unique. However, if we understand the ``uniqueness" following the philosophy in \ref{homocat}, it is ``unique", or precisely, $\text{Fun}(\Delta^2,C)\times_{\text{Fun}(\Lambda^2_1,C)}(g,\bullet,f)$ is contractible (\ref{composition_contractible_kan}).



\begin{remark}
    We denote the full subcategory of $\text{Set}_\Delta$ generated by quasi-categories as $\text{QCat}$.
\end{remark}


\begin{definition}
    We define $\text{Cat}_{\Delta}$ as the category of $\text{Set}_{\Delta}$-enriched categories, whose morphisms are given by enriched functors. We also define $\text{Cat}_{Kan}$ as the full subcategory of $\text{Cat}_{\Delta}$ generated by those Kan complex-enriched(which we will introduce in \ref{kan-complex}) categories.
\end{definition}


We will then show two model categories which model the $\infty$-categories, or more precisely, whose fibrant full subcatgoreis model the $\infty-$categories.


\begin{theorem}(Joyal, 2008)
    There exists a model structure called Joyal model structure on $\text{Set}_\Delta$ where fibrant objects are exactly quasi-categories.
\end{theorem}

\begin{theorem}(Bergner, 2004)
    There exists a model structure called Bergner model structure on $\text{Cat}_\Delta$, whose fibrant objects are exactly Kan complex-enriched categories.
    
\end{theorem}

Joyal and Tierney pointed out that these two model categories are Quillen equivalent, thus whose fibrant subcategoreis are also Quillen equivalent.

\begin{theorem}(Joyal \& Tierney, 2007)
    There exists a Quillen equivalence:

    $\mathfrak{C}:\text{Set}_\Delta^{Joyal} \rightleftarrows \text{Cat}_\Delta:\mathfrak{N}$.
\end{theorem}

\begin{corollary}
    $\text{QCat}$ and $\text{Cat}_{Kan}$ are Quillen equivalent .
\end{corollary}

\subsection{Mappsing Spaces}\label{mapping_space}

The idea of $\infty$-categories is that for any two objects in the category, there exists a ``mapping space" between such that the ``mapping space" is an $(\infty,0)$-category (Just as the case in $\mathcal{CG}$-enriched categories). That gives rise to the concept of Kan complexes where every morphism is ``invertible".

We still define the mapping space as the fibrant object in a homotoy category of a model category which we call as ``Kan-Quillen" model category. This model category is Quillen equivalent to $\text{Top}$, whose fibrant objects are CW complexes, that is, ``good spaces".





\begin{definition}\label{kan-complex}
    A simplical set X is called a Kan complex if it satisfies the right lifting property against horn inclusions:
\[
\begin{tikzcd}
\Lambda^n_k \arrow[r] \arrow[d, hook] & X \arrow[d] \\
\Delta^n \arrow[r] \arrow[ur, dashed, "\exists h" description] & \ast
\end{tikzcd}
\qquad(0 \leq k \leq n)
\]
\end{definition}

We make it clear why we say ``any morphism in Kan complex is invertible" here.

\begin{remark}
    Let $C$ be a quasi-category, the expression $x\in C$ means $x$ is a 0-simplex in $C$ (We also call $x$ as a 0-morphism); the expression $f:x\to y$ means $f$ is a 1-simplex in $C$ satisfying $d_0f=x,d_1f=y$ (We also call $f$ as a 1-morphism).
\end{remark}

\begin{theorem}(\textcite{ker}, 1.4.3.6)
    Let $C$ be a quasi-category, $x,y \in C, f,g:x\to y$, The following are equivalent:

    \begin{enumerate}
    \item There exists a 2-simplex $\sigma$ satisfying $d_o\sigma=id_y,d_1\sigma=g,d_2\sigma=f$,i.e.
\[
\begin{tikzcd}
& y \arrow[dr, "\text{id}_y"] & \\
x \arrow[ur, "f"] \arrow[rr, "g"'] & & y
\end{tikzcd}
\]
    
    \item There exists a 2-simplex $\sigma$ satisfying $d_o\sigma=f,d_1\sigma=g,d_2\sigma=id_x$,i.e.
\[
\begin{tikzcd}
& x \arrow[dr, "f"] & \\
x \arrow[ur, "\text{id}_x"] \arrow[rr, "g"'] & & y
\end{tikzcd}
\]
\end{enumerate}
\end{theorem}

\begin{remark}
    
Obviously the condition is a equivalence ralationship, we will say f is homotopic to g if one holds.

\end{remark}

\begin{definition}
    Let $C$ be an quasi-category, The homotopy category $\textsf{h}C$ of $C$ is defined as the ordinary category as below:

$\text{Ob}(\textsf{h}C)=\text{Ob}(C)$;

$\text{Hom}_{\textsf{h}C}(x,y)=\text{Hom}_C(x,y)/\sim$, where $\sim$ is defined by homotopy;

For $f:x\to y,g:y\to z \in \textsf{h}C$, the composition $g\circ f$ is defined as $d_1(h)$, where $h$ is a 2-morphism witenessing the composition of $f$ and $g$. This is well-defined because $\text{Fun}(\Delta^2,C)\times_{\text{Fun}(\Lambda^2_1,C)}(g,\bullet,f)$ is contractible.
    
\end{definition}

\begin{remark}
    Let $f:x\to y\in C$, we will call $f$ an isomorphism if the image $[f]$ of $f$ in $\textsf{h}C$ is an isomorphism.
\end{remark}

\begin{theorem}(\textcite{Joyal})
    A quasi-category $C$ is a Kan complex if and only if all morphisms in $C$ are isomorphisms.
\end{theorem}

Before formally giving the consturction of mapping space, we have to clairify the third model category whose fibrant objects are exactly Kan complexes. The following results are classic in algebraic topology.

\begin{theorem}
    There exists a model structure called Kan-Quillen model structure on $\text{Set}_\Delta$ where fibrant objects are Kan complexes. We call the weak equivalences in Kan-Quillen model as ``weak homotopy equivalences".
\end{theorem}

\begin{theorem}
    There exists a Quillen equivalence:

    $|\cdot|:\text{Set}_\Delta^{Kan-Quillen} \rightleftarrows \text{Top}:\text{Sing}$.
\end{theorem}

\begin{corollary}
    $\text{Kan}$ and $\text{CW}$ are Quillen equivalent.
\end{corollary}

It is well-known that CW complexes are ``good spaces", that is, whose ``mapping space" between two CW complexes itself is still a CW complex.



We will then formally give the consturction of mapping spaces.

\begin{definition}
    Let $X$ be a simplicial set, $X^{\lhd}:= \Delta^0 \star X;$ $ X^{\rhd}:= X \star \Delta^0$, where $\star$ represents join in simplicial set.
\end{definition}

\begin{definition}
    Let $X$ be a simplicial set, $x\in X$, the slice simplicial set $X_{/x}$ of $X$ over $x$ is defined by:

    $Hom(Y,X_{/x}):= Hom(Y^{\rhd},X)\underset{Hom(\Delta^0,X)}{\times}\{x\}$;

    the coslice simplicial set $X_{x/}$ of $X$ under $x$ is defined by:
    
    $Hom(Y,X_{x/}):= Hom(Y^{\lhd},X)\underset{Hom(\Delta^0,X)}{\times}\{x\}$;
\end{definition}

\begin{definition}
    Let $C$ be a quasi-category, $x,y\in C$. We will define:
    
    The left pinched complex $\text{Hom}_C^L(x,y):=C_{x/}\underset{C}{\times} \{y\}$;

    The right pinched complex $\text{Hom}_C^R(x,y):=\{x\}\underset{C}{\times} C_{/y}$;

    The balanced complex $\text{Hom}_C^B(x,y):=\{x\}\underset{\text{Fun}(\{0\},C)}{\times}\text{Fun}(\Delta^1,C))\underset{\text{Fun}(\{1\},C)}{\times}\{y\}$. 
\end{definition}

\begin{theorem}(\textcite{ker}, 4.6.5.5)\label{kan}
    $\text{Hom}_C^L(x,y)$, $\text{Hom}_C^R(x,y)$, and $\text{Hom}_C^B(x,y)$ are Kan complexes which are weak homotopy equivalent. 
\end{theorem}

As a result, we can well-defined the mapping space as a kan complex up to weak equivalence (in the sense of Kan-Quillen model structure).

\begin{definition}
    Let $C$ be a quasi-category, $x,y \in C$, we can define the mapping space $\text{Maps}(x,y)$ from $x$ to $y$ as the object in $\text{Set}_\Delta[W^{-1}]$ defined by \ref{kan}.
\end{definition}

In fact, $\text{Maps}$ can be extended to an $(\infty,1)-$functor $C^{op}\times C\to \mathscr{S}$, which we will show in \ref{maps}.

Based on this observation, we can illustrate the idea that mapping spaces are well-defined up to weak equivalance (following the philosophy in \ref{homocat}) by stating a stronger theorem in \ref{02GL}.


At the end of this subsection, we will make it clear what we mean by ``contractible".

\begin{definition}
    A Kan complex X is called contractible if it satisfies the right lifting property aganist boundary inclusions:
\[
\begin{tikzcd}
\partial \Delta^n \arrow[r] \arrow[d, hook] & X \arrow[d] \\
\Delta^n \arrow[r] \arrow[ur, dashed, "\exists h" description] & \ast
\end{tikzcd}
\] 
\end{definition}

\begin{theorem}(\textcite{ker}, 1.5.6.2)\label{composition_contractible_kan}
    Let $C$ be a quasi-category, $x,y,z\in C,f:x\to y,g: y\to z$, then $\text{Fun}(\Delta^2,C)\underset{\text{Fun}(\Lambda_1^2,C)}{\times}\{(g,\cdot,f)\}$ is a contractible Kan complex.
    
\end{theorem}

\subsection{Functors of $\infty$-categories}

Funtors between two quasi-categories are just morphisms in $\text{Set}_\Delta$. Two types of weak equivalences provide two types of ``equivalences". We'll show their meaning in the $(\infty,1)$-category of $(\infty,1)$-categories.



\begin{definition}
    Let $C,D$ be two quasi-categories, we will call the morphism $F:C\to D$ in $\text{Set}_\Delta$ as the functor from $C$ to $D$.
\end{definition}

\begin{proposition}
    $\text{Set}_\Delta$ admits a Catesian closed structure. With a little abuse of the notation, we will denote the corresponding internal hom $[X,Y]$ as $\text{Fun}(X,Y)\in \text{Set}_\Delta$. 
\end{proposition}

By Yoneda lemma clearly the 0-morphisms in $\text{Fun}(X,Y)$ are exactly the functors from $X$ to $Y$. We call this category the $(\infty,1)-$functor categories from $X$ to $Y$.

\begin{corollary}\label{nat_composition}
    Let $X,Y,Z$ be three simplicial sets, then there exists a natural composition $\text{Fun}(X,Y)\times \text{Fun}(Y,Z)\to \text{Fun}(X,Z)$.
\end{corollary}

\begin{theorem}(\textcite{lc}, 5.4.5)
    Let $C,D$ be two quasi-categories, $F:C\to D$ be a functor from $C$ to $D$. Then the following are equivalenct:

    1. $F$ is a weak equivalence in $\text{Set}_\Delta^{Joyal}$;

    2. For any quasi-category $D$, the induced morphism $\text{Fun}(D,E)\to \text{Fun}(C,E)$ is a weak equivalence in $\text{Set}_\Delta^{Joyal}$.

    We will call this weak equivalence as ``homotopy equivalence" or ``equivalence".
\end{theorem}

Following the philosophy in $\ref{homocat}$, this is the right ``isomorphism" in the $\infty-$cateogry of $\infty-$categories, that is, $\text{Cat}_\infty$.

\begin{lemma}(\textcite{ker}, 1.5.3.7)
    Let $X\in \text{Set}_\Delta,Y\in \text{QCat}$, then $\text{Fun}(X,Y)\in \text{QCat}$. 
\end{lemma}

\begin{lemma}(\textcite{ker}, 1.3.5.4)
    Let $X\in \text{QCat}$, then there exists a maximal Kan complex contained in $X$, which we denote as $X^\simeq$, or the core of $X$.
\end{lemma}

\begin{definition}
    Let $\vmathbb{QCat}$ be a $\text{Set}_\Delta-$enriched category which is defined as:
    
    Objects are all quasi-categories;

    $\text{Hom}(X,Y):=\text{Fun}(X,Y)^\simeq$;

    The composition map is given by $\ref{nat_composition}$.
\end{definition}

\begin{definition}
    We define the $(\infty,1)-$category of $(\infty,1)-$categories as $\mathfrak{N}(\vmathbb{QCat})$, which we denote as $\text{Cat}_\infty$.
\end{definition}

Similarily we have:

\begin{definition}
     Let $\vmathbb{Kan}$ be a $\text{Set}_\Delta-$enriched category which is defined as:
    
    Objects are all Kan complexes;

    $\text{Hom}(X,Y):=\text{Fun}(X,Y)^\simeq$;

    The composition map is given by $\ref{nat_composition}$.   
\end{definition}

\begin{definition}
    We define the $(\infty,1)-$category of $\infty-$groupoids as $\mathfrak{N}(\vmathbb{Kan})$, which we denote as $\text{Grpd}_\infty$ or $\mathscr{S}$.
\end{definition}

\begin{proposition}(\textcite{lc}, 5.7.6)
    The homotopy category $\textsf{h}\text{Cat}_\infty$ is equivalent to the ordinary category $\textsf{h}\vmathcal{QCat}$ defined as:

    $\text{Ob}(\textsf{h}\vmathcal{QCat}):= $ all small quasi-categories;

    $\text{Hom}_{\textsf{h}\vmathcal{QCat}}(C,D):=\pi_0(\text{Fun}(C,D)^\simeq)$;

    $[G]\circ[F]=[G\circ F]$.
\end{proposition}

\begin{theorem}(\textcite{lc}, 5.4.1)
    Let $C,D$ be two quasi-categories, $F:C\to D$, $F$ is a homotopy equivalence iff $[F]$ is an isomorphism in $\textsf{h}\vmathcal{QCat}$.
\end{theorem}

This explains the reasonibility of ``homotopy equivalence".




\section{Consturctions of $\infty$-categories}

\subsection{Fibrations of $\infty$-categories}
In ordinary category, we were already familiar with Grothendieck construction with respect to a kind of fibration. These things still have their correspondence in $\infty$-categories.

The relationship of those kinds of fibrations can be represented by the diagram below\cite{htt}.
\[
\begin{tikzcd}[row sep=large, column sep=large]
& \text{trivial fibration} \arrow[d] \\
& \text{Kan fibration} \arrow[dl] \arrow[dr] \\
\text{left fibration} \arrow[d] & & \text{right fibration} \arrow[d] \\
\text{coCartesian fibration} \arrow[dr] & & \text{Cartesian fibration} \arrow[dl] \\
& \text{categorical fibration} \arrow[d] \\
& \text{inner fibration}
\end{tikzcd}
\]

While the Grothendieck-Lurie construction\cite{htt} give the equivalence of $(\infty,1)-$categories:
\[
\begin{tabular}{c|c}

\textbf{The $\infty-$category of Fibrations} & \textbf{The $\infty$-category of Functors} \\

\hline

\text{LFib}_{/S} & \text{Fun}(S, \text{Grpd}_\infty) \\

\text{RFib}_{/S} & \text{Fun}(S^{\mathrm{op}}, \text{Grpd}_\infty) \\

\text{coCart}_{/S} & \text{Fun}(S, \text{Cat}_\infty) \\

\text{Cart}_{/S} & \text{Fun}(S^{\mathrm{op}}, \text{Cat}_\infty) \\

\end{tabular}
\]

In this subsection we will not illustrate everything in these two pictures. Instead, we only define left fibration and present the related Grothendieck-Luire construction to give a ``genuine" definition of $\text{Maps}$.


\begin{definition}
    Let $C,D$ be two quasi-cateogries, $p:C\to D$ is a functor, we say $p$ is an inner fibration if it satisfies the right lifting property against against inner horn inclustions:

\[
\begin{tikzcd}
\Lambda^n_k \arrow[r] \arrow[d, hook] & C \arrow[d] \\
\Delta^n \arrow[r] \arrow[ur, dashed, "\exists h" description] & D
\end{tikzcd}
\qquad(0 < k < n)
\]
\end{definition}

\begin{definition}
    Let $C,D$ be two $(\infty,1)-$categories, $p:C\to D$ be an inner fibration\footnote{As the word ``fibration" implies, inner fibration is stable under pullback, thus given $x\in C$ the fiber $C_{p(x)}$ is a quasi-category. In a model-independent theory this condition is not required.}. We say $p$ is a left fibration if it satisfies the right lifting property:

   \[
\begin{tikzcd}
\Lambda^n_k \arrow[r] \arrow[d, hook] & C \arrow[d] \\
\Delta^n \arrow[r] \arrow[ur, dashed, "\exists h" description] & D
\end{tikzcd}
\qquad (0\leq k < n)
\] 
\end{definition}


\begin{definition}
    Let $S$ be an $(\infty,1)-$category, we define the $(\infty,1)-$category $\text{LFib}_{/S}$ as the full subcategory of $\text{Cat}_\infty$ generated by those left fibrations.
\end{definition}

\begin{theorem}(\textcite{htt} 3.2.0.1)\label{3201}
    Given an $(\infty,1)$-category $S$, there exists an equivalence of $(\infty,1)$-categories:

    $\text{LFib}_{/S}\rightleftarrows\text{Fun}(S,\text{Grpd}_\infty)$.
\end{theorem}

Finally we can extend the definiton of $\text{Maps}$.

\begin{definition}
    Let $C$ be an $(\infty,1)-$category, we define the $(\infty,1)-$category of twisted arrow category $\text{TwArr}(C):=\text{Hom}_{\text{Set}_\Delta}(\text{N}_\bullet((-)^{op}\star(-)),C)$.
\end{definition}

\begin{proposition}(\textcite{ker}, 8.1.1.15)
    The projection $\text{TwArr}(C)\to C^{op}\times C$ induced by $[n]^{op}\to [n]^{op}\star [n]\leftarrow [n]$ is a left fibration.
\end{proposition}

Consider The projection as an object in $\text{LFib}_{/S}$, we can formly give our definition of $\text{Maps}$

\begin{definition}\label{maps}
    We define the functor $\text{Maps}:C^{op}\times C\to\text{Grpd}_\infty$ as the correspondence of the projection $\text{TwArr}(C)\to C^{op}\times C$ via \ref{3201}.
\end{definition}

We'll use the following definition in \ref{monoidal}.

\begin{definition}
    Let $C,D$ be two $(\infty,1)-$categories, $p:C\to D$ be an inner fibration, $f:x\to y\in C$. We say $f$ is $p$-coCartesian   if it satisfies the right lifting property:

   \[
\begin{tikzcd}
\Lambda^n_0 \arrow[r,"\phi"] \arrow[d, hook] & C \arrow[d] \\
\Delta^n \arrow[r] \arrow[ur, dashed, "\exists h" description] & D
\end{tikzcd}

\\ $n \geq 2, \phi|_{\{0,1\}}= f.$
\] 
\end{definition}

\subsection{Limits and Colimits}

\begin{definition}
    Let $C$ be an $(\infty,1)-$category, $x\in C$, we say $x$ is initial if $\forall y\in C$, $\text{Maps}(x,y)$ is contractible;  $x$ is final if $\forall y\in C$, $\text{Maps}(y,x)$ is contractible.
\end{definition}

\begin{definition}
    Let $K$ be a simplicial set, $C$ be an $(\infty,1)$-category, $u: K \to C$. The slice category of $C$ over $u$ is defined by:

    $\text{Hom}(Y,C_{/u}):= \text{Hom}(Y\star K,C)\underset{\text{Hom}(K,C)}{\times}\{u\}$;

    The coslice category of $C$ under $u$ is defined by:

    $\text{Hom}(Y,C_{u/}):= \text{Hom}(K\star Y,C)\underset{\text{Hom}(K,C)}{\times}\{u\}$.
    
\end{definition}




\begin{definition}
    Let $K$ be a simplicial set, $C$ be an $(\infty,1)$-category, $u: K \to C$. We say a diagram $\bar{u}:K^\triangleleft\to C$ is a limit diagram if it is a final object in $C_{/u}$; We say a diagram $\bar{u}:K^\triangleright\to C$ is a colimit diagram if it is an initial object in $C_{u/}$.

    We also say that $\bar{u} $ exhibits $\bar{u}(*)$ as a limit/colimit of $u$.
\end{definition}

\begin{definition}
    We say a limit diagram $\bar{u}:K^\triangleleft\to C$ is preserved by a functor $F:C\to C'$ if $F\circ \bar{u}:K^\triangleleft\to C'$ is a limit diagram; A colimit diagram $\bar{u}$ is preserved by $F$ if $F\circ\bar{u}$ is a colimit diagram.
\end{definition}

\begin{theorem}(\textcite{ker} 4.6.4.21)\label{02GL}
    Let $C,D$ be two $(\infty,1)-$categories, $F:C\to D$ be an equivalence, $u:K\to C$ be a diagram,  then the induced functor $C_{/u}\to C_{/F\circ u}$ is an equivalence.
\end{theorem}



\subsection{Adjoint functors}

The definition we give here is a slightly different from Lurie's classic one, which explicitly give the equivalence.

\begin{definition}
    Let $C,D$ be two $(\infty,1)$-categories, We say a pair of functors $F:C\rightleftarrows D:G$ is an adjunction pair if there exists an equivalence $\varphi:\text{Maps}(F(-),-)\to\text{Maps}(-,G(-))$ in $\text{Fun}(C^{op}\times D,\mathscr{S})$.

    We say that $F$ is a left adjoint functor which admits a right adjoint functor $G$, and vice versa.
\end{definition}

\begin{proposition}
    Let $F:C\rightleftarrows D:G$ be an adjunction pair, then $F$ preserves all small colimits while $G$ preserves all small limits.
\end{proposition}


\subsection{Presentable categories}

Roughly speaking, we will see a series of  correspondences between filtered version and normal version (just as the case in 1-category).

\begin{definition}
    Let $K$ be a simplicial set, $\kappa$ be a regular cardinal. We say $K$ is $\kappa$-small if $|S|<\kappa$, where $S$ is the set of nondegenerate simplexes in $K$.

    We say call a diagram $u:K\to C$ is $\kappa-$small if $K$ is $\kappa-$small.
\end{definition}

\begin{definition}
    Let $C$ be an $(\infty,1)-$category, we say $C$ is a $\kappa-$filtered category if any $\kappa$-small diagram $u:K\to C$ can be extended to $K^{\triangleright}\to C$.

    The corresponding limit/colimit diagram is called as $\kappa-$filtered limit/colimit diagram.
\end{definition}

\begin{definition}
    Let $C$ be an $(\infty,1)-$category, we define the presheaf category $\text{Psh}(C)$ of $C$ as $\text{Fun}(C^{op},\mathscr{S})$.

    We call the functor $h_C:C\to \text{PSh}(C)$, $X\mapsto \text{Maps}(-,X)$ as the ``Yoneda embedding".
\end{definition}

\begin{definition}
    Let $C$ be an $(\infty,1)$-category, we define the $\kappa$-ind category $\text{Ind}_\kappa(C)$ of $C$ as the full subcategory of $\text{PSh}(C)$ generated by theose objects which are small filtered colimits in $\text{PSh}(C)$ where $K\subset h_C(C)$.
\end{definition}

\begin{remark}
    Clearly $C\subset \text{Ind}_\kappa(C)$, thus we have another embedding functor from $C$ to $\text{Ind}_\kappa(C)$, which is denoted as $\iota$.
\end{remark}

We then show that Ind category is just the filtered version of presheaf category.


\begin{proposition}
    The Yoneda functor $h_C$ preserves small colimits while $\iota$ preserves $\kappa-$filtered small colimits.
\end{proposition}

Thus given a functor $F$ from $C$ to a cocomplete/$\kappa-$filtered cocomplete category $D$, it can be extended naturally to $\widetilde{F}:\text{PSh}(C)/\text{Ind}_\kappa(C)\to D$.

To be precise, that is:

\begin{theorem}
    There is an equivalence of category from $\text{LFun}(\text{PSh}(C),D)\simeq \text{Fun}(C,D)$, where $\text{LFun}(\text{PSh}(C),D)$ is the full subcategory of $\text{Fun}(\text{PSh}(C),D)$ generatied by those functors which preserve small colimits.
\end{theorem}

\begin{theorem}
    There is an equivalence of category from $\text{Fun}(\text{Ind}(C),D)_{\kappa-cont}\simeq \text{Fun}(C,D)$, where $\text{Fun}(\text{Ind}(C),D)_{\kappa-con}$ is the full subcategory of $\text{Fun}(\text{Ind}(C),D)$ generatied by those functors which preserve $\kappa$-small filtered colimits.
\end{theorem}

\begin{definition}
    Let $C$ be a $(\infty,1)-$category, $\kappa$ be a regular cardinal. We let $\operatorname{Ind}_\kappa(C)$ denote the full subcategory of $\text{Psh}(C)$ spanned by these functors $F:C^{op}\to \mathscr{S}$ which classifies all right Kan fibrations $f: \tilde{C}\to C$ where $\tilde{C} $ is $\kappa-$regular.
    
\end{definition}

\begin{definition}
    Let $C$ be a $(\infty,1)-$category, $\kappa$ be a regular cardinal. We way $C$ is $\kappa$-accesible if $\exists $ an essentially small $(\infty,1)-$ category $C_0$ such that

    $C\simeq \operatorname{Ind}_\kappa(C_0)$.

    We say $C$ is accessible if there exists a regular cardinal $\kappa$ such that $C$ is $\kappa-$accesible.

\end{definition}

\begin{definition}
    Let $C$ be a $(\infty,1)-$category. We way $C$ is locally presentable or presentable if $C$ satisfies:

    1. $C$ is accessible;

    2. $C$ admits all small colimits.
\end{definition}

There is an equivalent definition of presentbale categories, which may give a more intuitive understanding of the name.

\begin{definition}
    Let $C$ be an $\kappa-$accessible category, $D$ be a category. We say a functor $F:C\to D$ is $\kappa-$accessible if it presereves $\kappa-$filtered small colimits.

    We also say a functor $F$ is accessible if it is $\kappa-$accessible for some regular cardinal $\kappa$.
\end{definition}

\begin{theorem}
    An $(\infty,1)-$category $D$ is presentable iff there exsits an adjunction between $\text{PSh}(C)$ of an essentially small category $C$ and $D$:

    $F:\text{PSh}(C)\to D \dashv G:D\to\text{PSh}(C)$ such that

    $G$ is fully faithful and $G\circ F$ is an accessible functor.
\end{theorem}

\begin{example}
    Let $C$ be an $(\infty,1)$-category, then $\text{PSh}(C)$ is a presentable category.
\end{example}

We will see that $\text{Sh}(C)$ is also a presentable category with respect to any site.

\subsection{Reflective Localization}


\begin{definition}
    Let $C,D$ be two $(\infty,1)$-categories, $F:C\to D$, we say that $F$ is a reflective localization functor if:

    1. $F$ is a localization functor;

    2. $F$ admits a fully faithful right adjoint.

    We also say that $D$ is a reflective localization of $C$ if such $F$ exists.
\end{definition}

\begin{definition}
    Let $D$ be an $(\infty,1)-$category, $C$ be the full subcategory of $D$ with the inclusion functor $i:C\hookrightarrow D$. If $i$ admits a left adjoint, then we call $C$ the reflective subcategory of $D$.
\end{definition}

\begin{proposition}
    Let $C$ be the reflective subcategory of $D$, then the inclusion functor $i$ admits a reflective localization as the left adjoint.
\end{proposition}

\begin{proposition}
  Let $C$ be a complete, cocomplete $(\infty,1)$-category, $D$ be the reflective localization of $C$. Then $D$ is also complete and cocomplete.
\end{proposition}

\begin{proposition}
    Let $C$ be a presentbale $(\infty,1)$-category, $D$ be the reflective localization of $C$ via an accessible reflective localization functor $F:C\to D$. Then $D$ is also presentbale.
\end{proposition}


\begin{lemma}
    The sheaficiation functor is an accessible reflective localization functor.
\end{lemma}

\begin{example}
    Let $S$ be a site defeined over an $(\infty,1)$-category $C$, then the $(\infty,1)$-topos $\text{Sh}(C)$ is a presentable, cocomplete, complete category. 
\end{example}






\section{Stable $\infty$-categories}\label{stable}
 
\subsection{Stabilization}


\begin{definition}
    Let $C$ be an $(\infty,1)$-category, we say $C$ is a pointed category if $\exists $ zero object $0\in C$.
\end{definition}

\begin{theorem}
    Let $C$ be an $(\infty,1)$-category with a terminal object $*$, then the coslice category $C_{*/}$ is a pointed category. This process induces a functor which we denote as $()_+$.
\end{theorem}

\begin{definition}
    Let $C$ be a pointed $(\infty,1)$-category that admits finite colimits, $X,Y\in C,f: X\to Y$, then we define the cofiber $\text{cof}(f):=0 \sqcup_X Y$;
    
    If $C$ admits finite limits, we can define the fiber $\text{fib}(f):= 0\times_Y X$.
\end{definition}

\begin{definition}
    Let $C$ be a pointed $(\infty,1)$-category, a triangle is a diagram:

    \[
\begin{tikzcd}
X \ar[r,"f"] \ar[d] & Y \ar[d,"g"] \\
0 \ar[r] & Z
\end{tikzcd}
\]

We say the triangle is a fiber sequence if it is a pullback square, and a cofiber sequence if it is a pushout square.


\end{definition}

\begin{definition}
    Let $C$ be a pointed $(\infty,1)$-category, we say $C$ is stable if:

    1. $\forall f \in Mor(C)$, $f$ admits a fiber and a cofiber;

    2. A triangle is a fiber sequence if and only if it is a cofiber sequence.
\end{definition}



\begin{definition}
   Let $C$ be a pointed $(\infty,1)$-category that admits finite colimits, we can define the suspension functor $\Sigma:C\to C$ which maps

   $X\mapsto 0\sqcup_X0$;

   Let $C$ be a pointed $(\infty,1)$-category that admits finite limits, we can define the suspension functor $\Omega:C\to C$ which maps

   $X\mapsto 0\times_X0$.
\end{definition}

\begin{definition}
    $\operatorname{Pr}^{\mathrm{L}}$ is defined as the subcategory of $\text{Cat}_\infty$ whose objects are objects of $\text{Cat}_\infty$ while the morphisms are left adjoints;

    In duality, $\operatorname{Pr}^{\mathrm{R}}$ is defined as the subcategory of $\text{Cat}_\infty$ whose objects are objects of $\text{Cat}_\infty$ while the morphisms are right adjoints.
\end{definition}

\begin{proposition}\label{duality}
  Let $C$ be a pointed presentable $(\infty,1)$-category, then the $\operatorname{colim}[C \xrightarrow{\Sigma} C \xrightarrow{\Sigma} \cdots] \in \operatorname{Pr}^{\mathrm{L}}$ corresponds to $\operatorname{lim}[C \xleftarrow{\Omega} C \xleftarrow{\Omega} \cdots] \in \operatorname{Pr}^{\mathrm{R}}$ via $(\operatorname{Pr}^{\mathrm{L}})^{op}\simeq\operatorname{Pr}^{\mathrm{R}}$.

  We will denote the functor from $C$ to $\operatorname{Pr}^{\mathrm{L}}$ as $\operatorname{Sptr}$.
\end{proposition}

\begin{proposition}
    There is an adjunction pair:  $\Sigma: \mathrm{C} \rightleftarrows \mathrm{C}: \Omega$.
\end{proposition}

\begin{definition}
    We define $\Sigma^{\infty}$ to be the functor induced pointwise by stabiliztion $C\mapsto\operatorname{colim}[C \xrightarrow{\Sigma} C \xrightarrow{\Sigma} \cdots] $, while the $\Sigma^{\infty-k}$ denotes the composition of functor $\Sigma^\infty$ with the $\Sigma^{\infty}(\Sigma)$(Considering $\Sigma^\infty$ as the functor in $\text{Cat}_\infty$, the $\Omega^{\infty-k}$ denotes taking the $(k+1)$-th term of a spectrum.
\end{definition}


\begin{proposition}
    There is an adjunction pair:  $\Sigma^{\infty-k}: \mathrm{C} \rightleftarrows \text{Sptr}({C}): \Omega^{\infty-k}$.
\end{proposition}

\begin{proposition}
    Let $C$ be a pointed $(\infty,1)$-category that admits finite limits and colimits, then the following are equivalent:

    1. $C$ is stable
    
    2. The adjunction pair $\Sigma: \mathrm{C} \rightleftarrows \mathrm{C}: \Omega$ are equivalences.
    
    3. A square in $C$ is a pushout square if and only if it is a pullback square.
\end{proposition}

\subsection{$\infty-$Operads and Monoids}\label{monoidal}

\begin{definition}
      We denote the pointed finite set category as $\text{Fin}_*$. We denote $[n]\coprod \{*\}$ as $\langle n\rangle$.
\end{definition}

\begin{definition}
    Let $I\to J$ be a morphism in $\text{Fin}_*, \alpha:I\to J$. We say $\alpha$ is inert if $\forall j \in J\textbackslash \{*\}, \alpha^{-1}(j)$ is a singleton; We say $\alpha$ is active if $\alpha^{-1}(\{*\})=\{*\}$.
\end{definition}

\begin{definition}
    An $\infty-$operad $\mathcal{O}$ is a pair $(\mathcal{O}^\otimes,p)$ where $\mathcal{O}^\otimes$ is an $(\infty,1)-$category  and $p$ is a functor $\mathcal{O}^\otimes \to \text{Fin}_*$ such that(We denote $p^{-1}(\langle n\rangle)$ as $\mathcal{O}^\otimes_{\langle n\rangle}$):


    1. Any inert morphism $f: \langle m\rangle \to \langle n\rangle $ can be lifted to a $p$-coCartesian morphism $\tilde{f}$. In particular, the lifting induces the $(\infty,1)-$functor $f_!:\mathcal{O}^\otimes_{\langle m \rangle}\to \mathcal{O}^\otimes_{\langle n \rangle}$;

    2. The $(\infty,1)$-functor $\text{Maps}_f^\otimes(C_1,C_2)\to \prod_{1\leq i \leq n} \text{Maps}_{\rho^i\circ f}(C_1,C_3^i)$ induced by morphism $f: \langle m\rangle \to \langle n\rangle $ is an equivalence, where $\rho^i:\langle n \rangle \to \langle 1\rangle$ denotes the morphism which maps everthing but $i$ to $*$;

    3.For any $C_1,C_2,..., C_n\in \mathcal{O}^\otimes_{\langle 1 \rangle}$, there exists $C \in \mathcal{O}^\otimes_{\langle n \rangle}$ such that there exists a collection of the $p-$CoCartesian lifting from $C$ to $C_i$ of $\rho^i$.
\end{definition}


\begin{remark}
    In fact $\text{Fin}_*$ is an $\infty$-operad which we denote as $\text{Comm}^\otimes$.
\end{remark}

\begin{example}
    $(\text{Fin}_*,\textsf{Id})$ iteself is an $\infty-$operad. We denote it as $\text{Comm}^\otimes$ or $\mathbb{E}_\infty$.
\end{example}

\begin{example}
    Let $\text{Assoc}^\otimes$ denotes the subcategory of $\text{Fin}_*$ whose objects are those of $\text{Fin}_*$ but the morphisms are generated by those$f:\langle m \rangle \to \langle n \rangle$  such that for each $j\in \langle n \rangle \textbackslash \{*\},\alpha|_{\alpha^{-1}(j)}$ preserves ordering. 

    We also denote $(\text{Assoc}^\otimes,i)$ as $\mathbb{E}_1$.
\end{example}

\begin{definition}
    Let $(\mathcal{O}^\otimes,p)$ be an $\infty-$operad, we call a pair $(C^\otimes,p')$ where $C^\otimes$ is an $(\infty,1)-$category and $p'$ is a coCartesian fibration is an $\mathcal{O}-$monoidal $\infty-$category if the composite $p'\circ p$ exhibits $C^\otimes$ as an $\infty$-operad.
\end{definition}

\begin{remark}
    We will call the $\text{Comm}^\otimes$-monoidal categories as symmetric monoidal categories; We will call the $\text{Assoc}^\otimes-$monoidal categories as monoidal categories.
\end{remark}



\section{Six functor theories}

Six funtor theories, or six operations is a formalization of some normal operations in algebraic topology. It was Grothendieck who first introduced the theory, to explain $\acute{e}tale$ cohomology.

The contents of this section come from \textcite{ScholzeSF} and \textcite{HeyerMann2024}.

\begin{definition}(Naive)
    A six functor theory contains the following data:

(1) An $\infty$-category $C$ which admits finite limits.

(2) A class of morphisms $E$ contains all isomorphisms, and is stable under pullback and composition.

(3) An association $X \mapsto D(X)$ from $C$ to $\infty$- categories $D(X)$.

(4) A symmetric monoidal closed structure on $D(X)$, where the tensor functor and the internal hom functor are denoted as $\otimes$ and $\mathcal{H}om$.

(5) For each $f: X \rightarrow Y$, a pullback functor $f^*: D(Y) \rightarrow D(X)$ and a pushforward functor $f_*: D(X) \rightarrow D(Y)$.

(6) For each $f: X \rightarrow Y$ in $E$, a proper pullback functor $f^{!}: D(Y) \rightarrow D(X)$ and  a proper pullback functor $f^{!}: D(Y) \rightarrow D(X)$ .

which satisfies the following compatible conditions:



$f^*: D(Y) \rightarrow D(X) \quad \dashv \quad f_*: D(X) \rightarrow D(Y)$;

$f_{!}: D(X) \rightarrow D(Y) \quad \dashv \quad f^!: D(Y) \rightarrow D(X)$;

For a pullback diagram \[
\begin{tikzcd}
Y' \ar[r,"g'"] \ar[d,"f'"'] & Y \ar[d,"f"] \\
X' \ar[r,"g"] & X
\end{tikzcd}
\] 

there exists isormorphism of functors: $g^* f_{!} \simeq f_{!}^{\prime} g^{\prime *}$;

Let $f:Y\to X\in E, M\in D(X),N\in D(Y)$, then we have $M \otimes f_{!} N \simeq f_{!}\left(f^* M \otimes N\right)$.

$f^*$ and $f_!$ are compatible with composition.

\begin{definition}
    Let $(C,E)$ be a six functor theory, we can define the cohomology $\Gamma(X,\Lambda)=p_* 1$ for $X\in C$, whre $p:X \to *$, $1$ is the unit in $D(X)$.

    For $p \in E$, we can define the cohomology with compact support $\Gamma_c(X,\Lambda)=p_!1$.

    We define $\omega_X:=p^!1$, where $1$ is the unit of $D(*)$.
\end{definition}

\begin{proposition} (Küneth formula)

    Let $X,Y \in C$, then we have:

$    \Gamma_c(X \times Y, \Lambda) \simeq \Gamma_c(X, \Lambda) \otimes \Gamma_c(Y, \Lambda) .$
    
\end{proposition}

\begin{proposition}(Poincaré duality)

 $p_*\omega_X\simeq \mathcal{H}om(\Gamma_c(X,\Lambda),\Lambda)$.
    
\end{proposition}

\end{definition}
% !TEX root = ../main.tex
\chapter{Unstable Motivic Homotopy Theory}

The unstable motivic homotopy theory was first introduced by Voevodskey  \cite{Voevodsky1998} in 1998 to construct a homotopy theory in algebraic geometry. The basic intuition is that in such homotopy theory, the affine line $\mathbb{A}^1$ should play the same role as $\mathrm{I}$ in classic algebraic topology, so it was also called $\mathbb{A}^1$-homotopy theory.

In Voevodskey's original work, he used the simplicial sheaves to capture the higher information. However, as we've mentioned in the prelimaries, the development of $\infty$-categories supply a more convenient framework to capture them.

In this chapter, we will give the definition of motivic homotopy category $\mathcal{H}(S)$ related to a Noethrian scheme $S$, which plays the role of ``spaces" where algebraic topology works. With this intuition, we'll soon see the reasonability of its definition.

In short, the motivc homotopy category $\mathcal{H}(S)$ is just the homotopy localization at $\mathbb{A}^1$ of the $(\infty,1)$ presheaves over the Nisnevich site $Sm/S$, or equivalently, the homotopy localization at $\mathbb{A}^1$ of the $(\infty,1)$ presheaves over the Nisnevich site $Sm/S$ which satisfies excision property.

In the motivic homotopy category $\mathcal{H}(S)$, there are two trivial examples: the yoneda embedding from $Sm/S$ and the constant sheaf from $\mathscr{S}$. These two ``embedding" provide two kinds of spheres, i.e. the tate sphere and the standard sphere, thus inducing two types of stabilization in stable motivic homotopy theory.

    
\end{definition}
\section{Nisnevich topology}

In algebraic geometry, we were very familiar with two types of topology, i.e., the Zariski topology and the $\acute{e}$tale topology. However, the Zariksi topology is too coarse while the $\acute{e}$tale topology is too fine for algebtraic topology theories to work.

To be concrete, we can see the following two examples from \textcite{Hlavinka2022}.

\begin{example}
    Let $X\in Sm/\mathbb{C}$ with Zariski topology, $\mathcal{F}$ is a constant sheaf, then $\dim H^1(X(\mathbb{C}),\mathcal{F})= b_1(X\mathbb(\mathbb{C}))$ while $H^1(X,\mathcal{F})=0$, where $b_1$ denotes the first betti number.
\end{example}

%\begin{proof}
% ??? cohomology of constant sheaf= singular cohomology = betti number.    
%\end{proof}

\begin{example}
    Let $X= Spec(\mathbb{F}_q)$, then $\dim X =0$ while $H^1(X,\mathbb{Z}/p\mathbb{Z})\neq 0$, where $(p,q)=1$.
\end{example}

The right topology suitable for algebraic topology works ends up to be the Nisnevich topology, developed by Yevsey Nisnevich in 1989\cite{Nisnevich1989}.

\begin{definition}
    Given a Noethrian scheme $S$, consider in $Sm/S$, the category of smooth schemes of finite type over $S$,  a familiy of $\{\varphi_i: X_i \to Y\}_{i\in I}$ is called a Nisnevich cover if:

    1.$\varphi_i$ is $\acute{e}$tale;

    2.$\forall y\in Y, \exists i \in I, x\in \varphi_i^{-1}(y)$ such that the induced map of residule field $\widetilde{\varphi_i}: k(x)\to k(y)$ is an isomorphism.

    The collection of Nisnevich covers form a Grothendieck pretopology and the topology generated by it is called the Nisnevich topology. The site defined by it is called the Nisnevich site.
\end{definition}

\section{Excision property}

The so called ``excison property" can be understood in two intuitive ways, which motivate the two equivalent definitions below.

1. We expect the ``motivic spaces" should satisfy a kind of Mayer-Vietoris property.

2. Sheaves are presheaves which preserve colimits to limits, considering in stable cases a pullback square is also a pushout square.

\begin{definition}

Let $\mathcal{F}$ be a $(\infty,1)$-presheaf on the Nisnevich site $Sm/S$. We say $\mathcal{F}$ has the excision property if:

1.$\mathcal{F}(\emptyset)$ is contractible, i.e., isomorphic to the terminal object $*\in \mathscr{S}$ .

2.$\mathcal{F} $ satisfies one of the equivalent\cite{MorelVoevodsky1999} conditions below:
\begin{itemize}
    \item $\mathcal{F}$ is a sheaf;
    \item $\forall X\in Sm/S, U$ is a open subscheme of $X$, $X'\to X$ is a morphism in $Sm/S$, then 

    \begin{equation*}
\begin{tikzcd}[row sep=1.5em, column sep=1.5em]
\mathcal{F}(X) \arrow[r] \arrow[d] & \mathcal{F}(X') \arrow[d] \\
\mathcal{F}(U) \arrow[r] & \mathcal{F}(U \times_X X')
\end{tikzcd}
\end{equation*}

is a pullback square.
\end{itemize}


\end{definition}

\section{A1-equivalent}

As we've mentioned, the affine line $\mathbb{A}^1$ should play the role as $I$ in calssic algebraic topology world. We need every object is isomorphic to its ``cylinder object", thus we can define its ``left homotopy", as the case in model category.

\begin{remark}
    We will simply write $\mathbb{A}^1$ in replace of $\mathbb{A}^1_S$ in $Sm/S$.
\end{remark}

\begin{definition}
Let $\mathcal{F}$ be a $(\infty,1)$-presheaf on the Nisnevich site $Sm/S$. We say $\mathcal{F}$ is $\mathbb{A}^1-$equivalent if $\forall X\in Sm/S$, $\mathcal{F}(X)\to \mathcal{F}(X\times \mathbb{A}^1)$ is an isomorphism.
\end{definition}

The following theorem provides a functor called ``$\mathbb{A}^1$-localization functor", which make it possible to construct a map from $Sm/S$ to its motivic category.

\begin{theorem}\cite{morel2012a1} \label{localization}
    Let $\text{Sh}(Sm/S)$ be the category of $(\infty,1)$-sheaves on the Nisnevich site $Sm/S$, $\text{Sh}(Sm/S)_{\mathbb{A}^1}$ be the subcategory of $\text{Sh}(Sm/S)$ which satisfies $\mathbb{A}^1$-equivalent property, then the forgetful functor $U:\text{Sh}(Sm/S)_{\mathbb{A}^1}\to \text{Sh}(Sm/S)$ admits a right functor, which we denote as $L_{\mathbb{A}^1}$.
\end{theorem}

\begin{corollary}
    The category $\text{Sh}(Sm/S)_{\mathbb{A}^1}$ is a reflective sucategory of $\text{Sh}(Sm/S)$, with the reflective localization functor $L_{\mathbb{A}^1}$.
\end{corollary}

\begin{definition}
    Let $\mathcal{F},\mathcal{G} \in \text{Sh}(Sm/S)$,  $f,g: \mathcal{F}\to \mathcal{G}$, we say $f$ and $g$ are $\mathbb{A}^1$-homotopy via $h$ if $h: \mathcal{F}\times{\mathbb{A}^1\to \mathcal{G}}$ satisfies $h\circ i_0=f,h\circ i_1 =g$. This is denoted as $f\simeq_{\mathbb{A}^1}g$
\end{definition}
\[
\begin{tikzcd}
\mathcal{F} \ar[r, "i_0"] \ar[rd, "f"'] & \mathcal{F} \times \mathbb{A}^1 \ar[d, "h"] & \mathcal{F} \ar[l, "i_1"'] \ar[ld, "g"] \\
& \mathcal{G} &
\end{tikzcd}
\]

\begin{definition}
    Let $\mathcal{F},\mathcal{G} \in \text{Sh}(Sm/S)$, we say  $\mathcal{F}$ and $\mathcal{G}$ are $\mathbb{A}^1$-homotopy equivalent if $\exists f, g: \mathcal{F}\to \mathcal{G}$ such that $f\circ g \simeq_{\mathbb{A}^1} \text{id}_{\mathcal{G}}$ and $g\circ f \simeq_{\mathbb{A}^1} \text{id}_{\mathcal{F}}$.
\end{definition}

\begin{proposition}
  Let $\mathcal{F},\mathcal{G} \in \text{Sh}(Sm/S)$, $\mathcal{F}$ and $\mathcal{G}$ are $\mathbb{A}^1$-homotopy equivalent if and only if $L_{\mathbb{A}^1}(\mathcal{F})$ is isomorphic to $L_{\mathbb{A}^1}(\mathcal{G})$.
\end{proposition}


\section{Unstable motivic homotopy category}

After illustrating the reasonability of Nisnevich topology, excison property, $\mathbb{A}^1-$equivalent, we can finally give our definition of unstable motivic homotopy category.

\begin{definition}
    Let $S$ be a Noethrian scheme, the (unstable) motivic homotopy category $\mathcal{H}(S)$ is defined as the following equivalent category:
    
    1.The subcategory of $\text{Psh}(Sm/S)$ which satisfies excison property and $\mathbb{A}^1$-equivalent property;

    2.The subcategory of $\text{Sh}(Sm/S)$ which satisfies $\mathbb{A}^1$-equivalent property.


    We will call the objects in the motivic homotopy category as ``motivic spaces".

\end{definition}

We use presheaves of $Sm/S$ instead of $Sm/S$ because we expect the motivic homotopy category to be complete cocomplete and presentable, as $\mathscr{S}$. It is known that yoneda embedding is exactly the ``free cocompletion" and the presheave category is complete and presentable. (This idea also motivates the definition of algebraic spaces) After sheafification and $\mathbb{A}^1-$localization, this is still true because:

\begin{lemma}
    For any $(\infty,1)$-category $C$, its sheaf category $\text{Sh}(C)$ is a reflective subcategory of $\text{Psh}(C)$, with sheafification functor the reflective localization functor.
\end{lemma}

Since the reflective localization functor is stable under composition, we have:

\begin{proposition}
    $\mathcal{H}(S)$ is complete, cocomplete, and locally presentable.
\end{proposition}



Here are two trivial examples of objects in motivic homotopy category $\mathcal{H}(S)$.

\begin{example}
    Let $K\in \mathscr{S}$, then the constant sheaf over $Sm/S$ is a motivic space. We will still denote it as $K$.
\end{example}

\begin{example}
    Let $X/S\in Sm/S$, then the yoneda embedding of $X/S$ is a motivic space. With a little abuse of the notation, we still denote it as $X$.
\end{example}

% ??? proof: from definition.

These two examples provide two embedding from $\mathscr{S}$ and $Sm/S$ to $\mathcal{S}$. The two circles embedded from each category is what we call as ``standard sphere" and ``tate sphere".

\begin{definition}
    We call the constant sheaf in $\mathcal{H}(S)$ with value on $\Delta[1]/\partial\Delta[1]$ the standard sphere. 
\end{definition}

\section{Tate sphere}

We will talk in pointed case in this section. Since the functor $()_+$ has functorality in $Cat_\infty$, it commuates with any functor. Thus with an abuse of notation, we will not distinguish the theory in pointed case with that of unpointed case.

\begin{remark}
    We denote the product in $Sh_{\bullet}(Sm/S)$ as $\wedge$, i.e. the smash product.
\end{remark}

\begin{proposition}
    The following objects (or more precisely, the yoneda embedding of them) in $Sh_{\bullet}(Sm/S)$ are $\mathbb{A}^1$-equivalent.

    1. $\mathbb{P}^1$;

    2. cofiber of $\mathbb{G}_m\hookrightarrow \mathbb{A}^1$;

    3. $S^1\wedge \mathbb{G}_m$.
\end{proposition}

As a result, these objects in $Sh_{\bullet}(Sm/S)$ well-define an object in $\mathcal{H}(S)_\bullet$, which we call ``tate sphere" or ``motivic sphere".
    
As in classic algebraic topology, the ``sphere" induces the infinite loop functor, which gives rise to the stable motivic homotopy theory.
\chapter{Stable Motivic Homotopy Theory}

As we've shown in \ref{stable}, their is a standard process to get a stable $(\infty,1)$-category from a pointed $(\infty,1)$-category. The process happens in this chapter is a little different.

Instead of using the $(\infty,1)$-pushout of $0\gets X\to0$ (which concides with $X\wedge S^1$ in $\mathscr{S}_\bullet$), we use the original definition of suspension functor, which is just the smash product with ``circle".

However, we have two kinds of circles in motivic homotopy category, so correspondingly there are two types of suspension functors, which naturally lead to bispectrum as well as bicohomology.

The stabilization process gives rise to the stable motivic homotopy category $\mathcal{SH}(S)$. Robalo shows\cite{Robalo2015} the universal property of it, which show the reasonability of ``motivic" in its name.

\section{Stabilization}

This section is just a generalized version for biindexes of stabilzation process in higher algebra.

\begin{remark}
    Denote $S_t^1$ as the tate sphere, $S_s^1$ as the standard sphere.
\end{remark}

\begin{definition}
    The s-suspension functor $\Sigma_s:\mathcal{H}(S)_\bullet \to \mathcal{H}(S)_\bullet$ is defined as the functor: $-\wedge S_s^1$, while the t-suspension functor $\Sigma_t:\mathcal{H}(S)_\bullet \to \mathcal{H}(S)_\bullet$ is defined as the functor: $-\wedge S_t^1$.
\end{definition}

\begin{definition}
    The s-loop functor is defined as $\text{Maps}(S_s^1,-)$, while the t-loop functor is defined as $\text{Maps}(S_t^1,-)$.
\end{definition}

\begin{proposition}
    There are two adjunctions:

    $\Sigma_s: \mathcal{H}(S)_\bullet \rightleftarrows \mathcal{H}(S)_\bullet: \Omega_s$;

    $\Sigma_t: \mathcal{H}(S)_\bullet \rightleftarrows \mathcal{H}(S)_\bullet: \Omega_t$.
\end{proposition}


\begin{definition}\label{shs}
    The stable motivic homotopy category $\mathcal{SH}_s(S)$ (or simply denoted as $\mathcal{SH}(S)$) is defined as:

$    \operatorname{colim}[\mathcal{H}(S)\bullet \xrightarrow{\Sigma_s} \mathcal{H}(S)\bullet \xrightarrow{\Sigma_s} \cdots]\in \text{Pr}^L$

While the $\mathcal{SH}_t(S)$ can be defined as:

$    \operatorname{colim}[\mathcal{H}(S)\bullet \xrightarrow{\Sigma_t} \mathcal{H}(S)\bullet \xrightarrow{\Sigma_t} \cdots]\in \text{Pr}^L$

The object in motivic homotopy category is called ``motivic spectra".

\end{definition}

In duality, we can also define $    \operatorname{lim}[\mathcal{H}(S)\bullet \xleftarrow{\Omega_s} \mathcal{H}(S)\bullet \xleftarrow{\Omega_s} \cdots]\in \text{Pr}^R$, $    \operatorname{lim}[\mathcal{H}(S)\bullet \xleftarrow{\Omega_t} \mathcal{H}(S)\bullet \xleftarrow{\Omega_t} \cdots]\in \text{Pr}^R$. Since $\mathcal{H}(S)_\bullet$ is presentable, we have similar propositions like \ref{duality}.

\begin{proposition}

    The colimit $    \operatorname{colim}[\mathcal{H}(S)\bullet \xrightarrow{\Sigma_s} \mathcal{H}(S)\bullet \xrightarrow{\Sigma_s} \cdots]$ corresponds to the limit $\operatorname{lim}[\mathcal{H}(S)\bullet \xleftarrow{\Omega_s} \mathcal{H}(S)\bullet \xleftarrow{\Omega_s} \cdots]$ via $\text{Pr}^L\simeq (\text{Pr}^R)^{op}$;

    The colimit $    \operatorname{colim}[\mathcal{H}(S)\bullet \xrightarrow{\Sigma_t} \mathcal{H}(S)\bullet \xrightarrow{\Sigma_t} \cdots]$ corresponds to the limit $\operatorname{lim}[\mathcal{H}(S)\bullet \xleftarrow{\Omega_t} \mathcal{H}(S)\bullet \xleftarrow{\Omega_t} \cdots]$ via $\text{Pr}^L\simeq (\text{Pr}^R)^{op}$.
\end{proposition}

So we will not distingush the colimit and limit and call them all as $\mathcal{SH}_s(S)$ and $\mathcal{SH}(S)$.

\begin{remark}
    $\Sigma_{*}:\mathcal{H}(S)_\bullet\to \mathcal{SH}_{*}(S)$ and $\Omega_{*}:\mathcal{H}(S)_\bullet\to \mathcal{SH}_{*}(S)$ each natrually induce functor from $\mathcal{SH}_{*}(S)$ to $\mathcal{SH}_{*}(S)$ $(*\in \{s,t\})$. With a little abuse of notation, we still denote them as $\Sigma_{*} $ and $\Omega_{*}$.
\end{remark}


\begin{remark}
    
Since the stabilization precess we've mentioned in \ref{shs} has functorality, the process can be considered as a functor from $\mathcal{H}(S)_\bullet$ to $\mathcal{SH}(S)$, which we denoted as $\Sigma_*^\infty$, $*\in\{s,t\}$.

Duality, we denote the functor from $\mathcal{SH}(S)$  to $\mathcal{H}(S)_\bullet$ which maps a spectrum to its first term.
\end{remark}

\begin{remark}
    We will denote$\Sigma_*[k]$  as $\Sigma_*^k$, $\Omega_*[k]$ as $\Omega_*^k,*\in\{s,t\}$;

    We will denote $\Sigma_*^\infty\circ\Sigma_*[k]$  as $\Sigma_*^{\infty-k}$, $\Omega_*^\infty\circ\Omega_*[k]$  as $\Omega_*^{\infty-k},*\in\{s,t\}$.

    It's easy to verify the commutativity of these functors, which is why we use additivity to represent its index.
\end{remark}

\begin{proposition}
    There are two adjunctions:

$    \Sigma_*^{\infty-k}: \mathcal{H}(S)\bullet \rightleftarrows \mathcal{SH}_*(S): \Omega_*^{\infty-k}$ , $*\in \{s,t\}$;
\end{proposition}

As for bigraded cases, the situation become a little more complicated.

\begin{definition}
The category $\mathcal{SH}_{s,t}(S)$ can be defined as:
\[
\mathop{\mathrm{lim}}\limits_{\substack{\longrightarrow \\ \Sigma_s, \Sigma_t}} 
\left[
\begin{tikzcd}[
  row sep=1.5em, 
  column sep=1.5em,
  cells={nodes={minimum width=2em}} % 统一节点宽度
]
\cdots \arrow[r, "\Sigma_s"] & \cdots \arrow[r, "\Sigma_s"] & \cdots \arrow[r, "\Sigma_s"] & \cdots \\
\mathcal{H}(S)_\bullet \arrow[r, "\Sigma_s"] \arrow[u, "\Sigma_t"'] & 
\mathcal{H}(S)_\bullet \arrow[r, "\Sigma_s"] \arrow[u, "\Sigma_t"'] & 
\mathcal{H}(S)_\bullet \arrow[r, "\Sigma_s"] \arrow[u, "\Sigma_t"'] & 
\cdots \arrow[u, "\Sigma_t"'] \\
\mathcal{H}(S)_\bullet \arrow[r, "\Sigma_s"] \arrow[u, "\Sigma_t"'] & 
\mathcal{H}(S)_\bullet \arrow[r, "\Sigma_s"] \arrow[u, "\Sigma_t"'] & 
\mathcal{H}(S)_\bullet \arrow[r, "\Sigma_s"] \arrow[u, "\Sigma_t"'] & 
\cdots \arrow[u, "\Sigma_t"'] \\
\mathcal{H}(S)_\bullet \arrow[r, "\Sigma_s"] \arrow[u, "\Sigma_t"'] & 
\mathcal{H}(S)_\bullet \arrow[r, "\Sigma_s"] \arrow[u, "\Sigma_t"'] & 
\mathcal{H}(S)_\bullet \arrow[r, "\Sigma_s"] \arrow[u, "\Sigma_t"'] & 
\cdots \arrow[u, "\Sigma_t"']
\end{tikzcd}
\right]\in \text{Pr}^L

\]

\end{definition}

Similarily we can define $  \Sigma_{s,t}^{\infty,\infty}$ as the stabilization functor and $\Omega_{s,t}^{\infty,\infty}$ as taking the $(1,1)-$term of the spectrum. The definition of $\Sigma_{s,t}^{i,j}$ as well as $\Omega_{s,t}^{i,j}$ is similar.

\begin{proposition}
    $    \Sigma_{s,t}^{\infty-i,\infty-j}: \mathcal{H}(S)\bullet \rightleftarrows \mathcal{SH}_{s,t}(S): \Omega_{s,t}^{\infty-i,\infty-j}$.
\end{proposition}

\section{Motivic cohomology}

In classic algebraic topology, Brown representation theorem tell us that there exists a 1-1 corresponding between cohomology theories and spectra. Naumann and Spitzweck proved\cite{Naumann2009BrownRI} that the theorem still holds in motivic world, after giving the axioms which a bigraded cohomology theory should satisfy.

\begin{definition}
    We call a sequnce of functor $(F^{p,q})_{(p,q\in \mathbb{Z}_{\geq 0}^2)}:(Sm/S)^{op}\to \mathcal{E}ns$ is a motivic cohomology if:

    1. $F^{p+1,q}(S_s^1\wedge X)= F^{p,q}(X)$;

    2. $F^{p,q+1}(S_t^1\wedge X)= F^{p,q}(X)$;

    3. $F^{p,q}$ satisfies Nisnevich excision property and $\mathbb{A}^1-$invariant property. 
\end{definition}

\begin{theorem}
    A motivic cohomology is represented by a motivic spectrum $E$ by $F^{p,q}(X):=[X,\Omega_{s,t}^{\infty-(p-2q),\infty-q}E]=[\Sigma_{s,t}^{\infty}X,\Omega_{s,t}^{p-2q,q}E]$, and vice versa.
\end{theorem}

We will show some classic examples of motivic spectra together with their corresponding motivic cohomology.

\section{Motivic Spectra}

We firstly see the motivic Eilenberg-MacLane spectrum. The construction below come from \textcite{dundas2007motivic}.

\begin{definition}
    Let $X\in Sm/S$, define $L(X):(Sm/S)^{op}\to Ab$ which maps:

    $U\mapsto F(Irr(U,X))$,
    
    where $Irr(U,X)$ is the set of closed irreducible subscheme $Y$ of $U\times X$ such that $p_0:Y\to U$ is of finite type and $im(p_0)$ covers one of the connected compnents of $U$, $F$ is the free abelization functor.
\end{definition}

\begin{proposition}
    The functor $U\circ L$ is a Nisnevich sheaf which satisfies $\mathbb{A}^1-$equivalent category, where $U$ is the forgetful functor. Thus $U\circ L\in \mathcal{H}(S)$.
\end{proposition}

\begin{definition}
    The Eilenberg-MacLane spectrum $\mathbf{H}\mathbb{Z}^{p,q}:=  L\left((S_s^1)^{\wedge p} \wedge (S_t^1)^{\wedge q}\right)$.

\end{definition}

We don't forget its group structure here because we want to tansfer the cohomology theory.

Just as in classic algebraic topology, Eilenberg-MacLane spectrum represents the ``singular cohomology", that is, the Bloch's higher Chow group.

\begin{proposition}
    $[X,\mathbf{H}\mathbb{Z}^{p,q}]= CH^q(X,2q-p)$.
\end{proposition}

The next example we're cared about is the algebaric K-theory spectrum.

\begin{definition}
    Define $\text{BGL}_n:=\operatorname{colim} (...\to\text(Gr_n(\mathbb{A}^{n+m}))\to \text(Gr_n(\mathbb{A}^{n+m+1}))\to...)$;

    $\text{KGL}_n:= \mathbb{Z}\times \text{BGL}_n$.

    $\text{KGL}:=\operatorname{colim}[...\to\text{KGL}_{n-1}\to \text{KGL}_n\to \text{KGL}_{n+1}\to...]$
\end{definition}

\begin{proposition}
    There exists $\beta: \mathbb{P}^1\wedge \text{KGL}\to \text{KGL}$.
\end{proposition}

\begin{definition}
    We call the spectrum $(\text{KGL},\text{KGL},...)$ together with $\beta$ on each degree the algebraic K-theory spectrum.
\end{definition}

\begin{proposition}
    The algebraic K-theory spectrum represents algebraic K-theory.
\end{proposition}

\section{Universal Property}

In this section, we will show that the stable motivic homotopy category $\mathcal{SH}(S)$ is indeed ``motivic",i.e. has the universal property, which is proved by Robalo in 2012\cite{Robalo2012arxiv}.

\begin{definition}
    Let $X,Y \in \mathcal{SH}(S),X \otimes Y:=\lim _n n \lim _m\left(\Sigma^{\infty}\left(X_n \wedge Y_m\right)[-(n+m)]\right)$
\end{definition}

\begin{definition}
    Let $X,Y \in \mathcal{H}(S), map(X,Y):=([X,Y[n]],[[X,Y[n]]\to [X,\Omega Y[n+1]]\to \Omega[X,Y[n+1]]])$.
\end{definition}

\begin{proposition}
    The $\otimes$ functor and the $map$ functor define a symmetric monoidal closed structure on $\mathcal{SH}(S)$.
\end{proposition}

Since $\mathcal{H}(S)$ is presentable, we have:

\begin{proposition}
    $\mathcal{SH}(S)$ is pointed and presentable.
\end{proposition}

\begin{proposition}
    The composition of functor $L_{\mathbb{A}^1}\circ()_+\circ h_C$: $Sm/S \to \mathcal{SH}(S)$ is a symmetric monoidal, colimit-preserving functor.
\end{proposition}

\begin{theorem}\cite{Robalo2012arxiv}
    Let $D^{\otimes}$ be a pointed presentable symmetric monoidal $(\infty,1)-$category, then the composition map:

    $Fun^{\otimes,L}(\mathcal{SH}(S)^\otimes,D^\otimes)\to Fun^{\otimes,L}(Sm/S^\times,D^\otimes)$ 

    is fully faithful and its image satisfies:

    1.Nisnevich excision property;

    2.$\mathbb{A}^1$-invariant; 

    3.send Tate sphere to an invertible object.

    
\end{theorem}

This theorem completes the original motivation for our topic, i.e. find a universal cohomology theory such that every cohomology theory passes through it.

\section{Six functor fomalism}

In 2007, Ayoub \cite{Ayoub2007} found that the stable motivic homotopy category $\mathcal{SH}(S)$ admits a six functor theory, which can be seen as an enhance of cohomology theory.

\begin{theorem}\cite{Ayoub2007}
    There is a six functor theory on $(Sm/S,\mathcal{SH}(S))$, where $E$ is the set of separated morphisms of finite type.
\end{theorem}


}

%TC:ignore

\clearpage
{
\ExplSyntaxOn
\bool_if:NTF \g__sjtu_twoside_bool
{
    \fancyhead [ LE ]     { References }
    \fancyhead [ RO ]     { References }
}
{
    \fancyhead [ R ] { References }
}
\ExplSyntaxOff
% 文献表字体
\renewcommand{\bibfont}{\zihao{5}}
% 设定固定间距
\fixedlineskip{15.6bp}
{
\ctexset{chapter={afterskip=26bp}}
% 参考文献
\printbibliography[heading=bibintoc]
}
\clearpage
}

\makeatletter
% \appendix采用数字编号。
\renewcommand{\appendix}{\par
    \setcounter{chapter}{0}
    \setcounter{section}{0}
    \ctexset{chapter/number={\arabic{chapter}}}
}
% 使用 \appchapter 替代附录中的 \chapter 章节,附录中的章节不再放入目录。
\newcommand{\appchapter}[1]{
    \refstepcounter{chapter}
    \SJTU@head*[Appendix  \thechapter]{#1 Appendix  \thechapter}
}
\makeatother

{
\ctexset{chapter={afterskip=26bp}}
% 附录
\appendix
% 附录中图表不加入索引
\captionsetup{list=no}
% !TEX root = ../main.tex

\appchapter{Symbols and Marks }\label{chap:symbol}
\addcontentsline{toc}{chapter}{Appendix}  

\begin{itemize}

\item $\star$ - the join operation of simplicial set

\item $h_C$ - the yoneda embedding

\item $\text{Maps}(-,-)$ - the mapping space of two $(\infty,1)$-categories
\item $\mathscr{S}$ - the simplical nerve of Kan complex enriched category, which models the category of infinite groupoid
\item $\mathcal{E}ns$ - the category of set

\item $\otimes,\mathcal{H}om$ - the tensor functor and internel Hom functor for a monoidal symmetric closed structure

\item $\text{Cat}_\infty$ - the simplicial nerve of the simplical enriched category, which models the $(\infty,1)-$category of $(\infty,1)$-categories

\item $Y/X$ - the (homotopy) cofiber of the morphism $f: X\to Y$
\end{itemize}
}


% 结尾部分
\backmatter

\clearpage
{
\ExplSyntaxOn
\bool_if:NTF \g__sjtu_twoside_bool
{
    \fancyhead [ LE ]     { Research \ Projects \ and \ Publications }
    \fancyhead [ RO ]     { Research \ Projects \ and \ Publications}
}
{
    \fancyhead [ R ] { Research \ Projects \ and \ Publications }
}
\ExplSyntaxOff
{
\ctexset{chapter={afterskip=26bp}}
% 发表论文及科研成果
}
\clearpage
}

\clearpage
{
\ExplSyntaxOn
\bool_if:NTF \g__sjtu_twoside_bool
{
    \fancyhead [ LE ]     { Acknowledgements }
    \fancyhead [ RO ]     { Acknowledgements }
}
{
    \fancyhead [ R ] { Acknowledgements }
}
\ExplSyntaxOff
{
\ctexset{chapter={afterskip=26bp}}
% 致谢
% !TEX root = ../main.tex

\begin{acknowledgements}
I would like to express my deepest gratitude to those who have supported me throughout this research journey.

First, I am grateful to my advisor, Professor Yong Hu, for his guidance. Beyond his academic insights, Professor Hu provided invaluable administrative support that significantly facilitated my research process. 

My sincere thanks go to my fellow participants in the seminar series over the past year, particularly Hairuo Xiao. Our regular discussions and his penetrating insights into algebraic geometry have substantially enriched my understanding of the subject. The collaborative environment we maintained in our study group has been both intellectually stimulating and personally rewarding.

Lastly, I owe immeasurable gratitude to my parents. Their unwavering support - both emotional and financial - has been the foundation upon which I could pursue my academic interests with freedom and dedication.

To all these individuals, I extend my heartfelt appreciation for making this academic endeavor possible.


\end{acknowledgements}

}
\clearpage
}

{
\ctexset{chapter={afterskip=26bp}}
% 学士学位论文要求在最后有一个大摘要,单独编页码
%\input{contents/digest}
}

\end{document}
