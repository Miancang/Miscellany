% !TEX root = ../main.tex
\chapter{Unstable Motivic Homotopy Theory}

The unstable motivic homotopy theory was first introduced by Voevodskey  \cite{Voevodsky1998} in 1998 to construct a homotopy theory in algebraic geometry. The basic intuition is that in such homotopy theory, the affine line $\mathbb{A}^1$ should play the same role as $\mathrm{I}$ in classic algebraic topology, so it was also called $\mathbb{A}^1$-homotopy theory.

In Voevodskey's original work, he used the simplicial sheaves to capture the higher information. However, as we've mentioned in the prelimaries, the development of $\infty$-categories supply a more convenient framework to capture them.

In this chapter, we will give the definition of motivic homotopy category $\mathcal{H}(S)$ related to a Noethrian scheme $S$, which plays the role of ``spaces" where algebraic topology works. With this intuition, we'll soon see the reasonability of its definition.

In short, the motivc homotopy category $\mathcal{H}(S)$ is just the homotopy localization at $\mathbb{A}^1$ of the $(\infty,1)$ presheaves over the Nisnevich site $Sm/S$, or equivalently, the homotopy localization at $\mathbb{A}^1$ of the $(\infty,1)$ presheaves over the Nisnevich site $Sm/S$ which satisfies excision property.

In the motivic homotopy category $\mathcal{H}(S)$, there are two trivial examples: the yoneda embedding from $Sm/S$ and the constant sheaf from $\mathscr{S}$. These two ``embedding" provide two kinds of spheres, i.e. the tate sphere and the standard sphere, thus inducing two types of stabilization in stable motivic homotopy theory.

    
\end{definition}
\section{Nisnevich topology}

In algebraic geometry, we were very familiar with two types of topology, i.e., the Zariski topology and the $\acute{e}$tale topology. However, the Zariksi topology is too coarse while the $\acute{e}$tale topology is too fine for algebtraic topology theories to work.

To be concrete, we can see the following two examples from \textcite{Hlavinka2022}.

\begin{example}
    Let $X\in Sm/\mathbb{C}$ with Zariski topology, $\mathcal{F}$ is a constant sheaf, then $\dim H^1(X(\mathbb{C}),\mathcal{F})= b_1(X\mathbb(\mathbb{C}))$ while $H^1(X,\mathcal{F})=0$, where $b_1$ denotes the first betti number.
\end{example}

%\begin{proof}
% ??? cohomology of constant sheaf= singular cohomology = betti number.    
%\end{proof}

\begin{example}
    Let $X= Spec(\mathbb{F}_q)$, then $\dim X =0$ while $H^1(X,\mathbb{Z}/p\mathbb{Z})\neq 0$, where $(p,q)=1$.
\end{example}

The right topology suitable for algebraic topology works ends up to be the Nisnevich topology, developed by Yevsey Nisnevich in 1989\cite{Nisnevich1989}.

\begin{definition}
    Given a Noethrian scheme $S$, consider in $Sm/S$, the category of smooth schemes of finite type over $S$,  a familiy of $\{\varphi_i: X_i \to Y\}_{i\in I}$ is called a Nisnevich cover if:

    1.$\varphi_i$ is $\acute{e}$tale;

    2.$\forall y\in Y, \exists i \in I, x\in \varphi_i^{-1}(y)$ such that the induced map of residule field $\widetilde{\varphi_i}: k(x)\to k(y)$ is an isomorphism.

    The collection of Nisnevich covers form a Grothendieck pretopology and the topology generated by it is called the Nisnevich topology. The site defined by it is called the Nisnevich site.
\end{definition}

\section{Excision property}

The so called ``excison property" can be understood in two intuitive ways, which motivate the two equivalent definitions below.

1. We expect the ``motivic spaces" should satisfy a kind of Mayer-Vietoris property.

2. Sheaves are presheaves which preserve colimits to limits, considering in stable cases a pullback square is also a pushout square.

\begin{definition}

Let $\mathcal{F}$ be a $(\infty,1)$-presheaf on the Nisnevich site $Sm/S$. We say $\mathcal{F}$ has the excision property if:

1.$\mathcal{F}(\emptyset)$ is contractible, i.e., isomorphic to the terminal object $*\in \mathscr{S}$ .

2.$\mathcal{F} $ satisfies one of the equivalent\cite{MorelVoevodsky1999} conditions below:
\begin{itemize}
    \item $\mathcal{F}$ is a sheaf;
    \item $\forall X\in Sm/S, U$ is a open subscheme of $X$, $X'\to X$ is a morphism in $Sm/S$, then 

    \begin{equation*}
\begin{tikzcd}[row sep=1.5em, column sep=1.5em]
\mathcal{F}(X) \arrow[r] \arrow[d] & \mathcal{F}(X') \arrow[d] \\
\mathcal{F}(U) \arrow[r] & \mathcal{F}(U \times_X X')
\end{tikzcd}
\end{equation*}

is a pullback square.
\end{itemize}


\end{definition}

\section{A1-equivalent}

As we've mentioned, the affine line $\mathbb{A}^1$ should play the role as $I$ in calssic algebraic topology world. We need every object is isomorphic to its ``cylinder object", thus we can define its ``left homotopy", as the case in model category.

\begin{remark}
    We will simply write $\mathbb{A}^1$ in replace of $\mathbb{A}^1_S$ in $Sm/S$.
\end{remark}

\begin{definition}
Let $\mathcal{F}$ be a $(\infty,1)$-presheaf on the Nisnevich site $Sm/S$. We say $\mathcal{F}$ is $\mathbb{A}^1-$equivalent if $\forall X\in Sm/S$, $\mathcal{F}(X)\to \mathcal{F}(X\times \mathbb{A}^1)$ is an isomorphism.
\end{definition}

The following theorem provides a functor called ``$\mathbb{A}^1$-localization functor", which make it possible to construct a map from $Sm/S$ to its motivic category.

\begin{theorem}\cite{morel2012a1} \label{localization}
    Let $\text{Sh}(Sm/S)$ be the category of $(\infty,1)$-sheaves on the Nisnevich site $Sm/S$, $\text{Sh}(Sm/S)_{\mathbb{A}^1}$ be the subcategory of $\text{Sh}(Sm/S)$ which satisfies $\mathbb{A}^1$-equivalent property, then the forgetful functor $U:\text{Sh}(Sm/S)_{\mathbb{A}^1}\to \text{Sh}(Sm/S)$ admits a right functor, which we denote as $L_{\mathbb{A}^1}$.
\end{theorem}

\begin{corollary}
    The category $\text{Sh}(Sm/S)_{\mathbb{A}^1}$ is a reflective sucategory of $\text{Sh}(Sm/S)$, with the reflective localization functor $L_{\mathbb{A}^1}$.
\end{corollary}

\begin{definition}
    Let $\mathcal{F},\mathcal{G} \in \text{Sh}(Sm/S)$,  $f,g: \mathcal{F}\to \mathcal{G}$, we say $f$ and $g$ are $\mathbb{A}^1$-homotopy via $h$ if $h: \mathcal{F}\times{\mathbb{A}^1\to \mathcal{G}}$ satisfies $h\circ i_0=f,h\circ i_1 =g$. This is denoted as $f\simeq_{\mathbb{A}^1}g$
\end{definition}
\[
\begin{tikzcd}
\mathcal{F} \ar[r, "i_0"] \ar[rd, "f"'] & \mathcal{F} \times \mathbb{A}^1 \ar[d, "h"] & \mathcal{F} \ar[l, "i_1"'] \ar[ld, "g"] \\
& \mathcal{G} &
\end{tikzcd}
\]

\begin{definition}
    Let $\mathcal{F},\mathcal{G} \in \text{Sh}(Sm/S)$, we say  $\mathcal{F}$ and $\mathcal{G}$ are $\mathbb{A}^1$-homotopy equivalent if $\exists f, g: \mathcal{F}\to \mathcal{G}$ such that $f\circ g \simeq_{\mathbb{A}^1} \text{id}_{\mathcal{G}}$ and $g\circ f \simeq_{\mathbb{A}^1} \text{id}_{\mathcal{F}}$.
\end{definition}

\begin{proposition}
  Let $\mathcal{F},\mathcal{G} \in \text{Sh}(Sm/S)$, $\mathcal{F}$ and $\mathcal{G}$ are $\mathbb{A}^1$-homotopy equivalent if and only if $L_{\mathbb{A}^1}(\mathcal{F})$ is isomorphic to $L_{\mathbb{A}^1}(\mathcal{G})$.
\end{proposition}


\section{Unstable motivic homotopy category}

After illustrating the reasonability of Nisnevich topology, excison property, $\mathbb{A}^1-$equivalent, we can finally give our definition of unstable motivic homotopy category.

\begin{definition}
    Let $S$ be a Noethrian scheme, the (unstable) motivic homotopy category $\mathcal{H}(S)$ is defined as the following equivalent category:
    
    1.The subcategory of $\text{Psh}(Sm/S)$ which satisfies excison property and $\mathbb{A}^1$-equivalent property;

    2.The subcategory of $\text{Sh}(Sm/S)$ which satisfies $\mathbb{A}^1$-equivalent property.


    We will call the objects in the motivic homotopy category as ``motivic spaces".

\end{definition}

We use presheaves of $Sm/S$ instead of $Sm/S$ because we expect the motivic homotopy category to be complete cocomplete and presentable, as $\mathscr{S}$. It is known that yoneda embedding is exactly the ``free cocompletion" and the presheave category is complete and presentable. (This idea also motivates the definition of algebraic spaces) After sheafification and $\mathbb{A}^1-$localization, this is still true because:

\begin{lemma}
    For any $(\infty,1)$-category $C$, its sheaf category $\text{Sh}(C)$ is a reflective subcategory of $\text{Psh}(C)$, with sheafification functor the reflective localization functor.
\end{lemma}

Since the reflective localization functor is stable under composition, we have:

\begin{proposition}
    $\mathcal{H}(S)$ is complete, cocomplete, and locally presentable.
\end{proposition}



Here are two trivial examples of objects in motivic homotopy category $\mathcal{H}(S)$.

\begin{example}
    Let $K\in \mathscr{S}$, then the constant sheaf over $Sm/S$ is a motivic space. We will still denote it as $K$.
\end{example}

\begin{example}
    Let $X/S\in Sm/S$, then the yoneda embedding of $X/S$ is a motivic space. With a little abuse of the notation, we still denote it as $X$.
\end{example}

% ??? proof: from definition.

These two examples provide two embedding from $\mathscr{S}$ and $Sm/S$ to $\mathcal{S}$. The two circles embedded from each category is what we call as ``standard sphere" and ``tate sphere".

\begin{definition}
    We call the constant sheaf in $\mathcal{H}(S)$ with value on $\Delta[1]/\partial\Delta[1]$ the standard sphere. 
\end{definition}

\section{Tate sphere}

We will talk in pointed case in this section. Since the functor $()_+$ has functorality in $Cat_\infty$, it commuates with any functor. Thus with an abuse of notation, we will not distinguish the theory in pointed case with that of unpointed case.

\begin{remark}
    We denote the product in $Sh_{\bullet}(Sm/S)$ as $\wedge$, i.e. the smash product.
\end{remark}

\begin{proposition}
    The following objects (or more precisely, the yoneda embedding of them) in $Sh_{\bullet}(Sm/S)$ are $\mathbb{A}^1$-equivalent.

    1. $\mathbb{P}^1$;

    2. cofiber of $\mathbb{G}_m\hookrightarrow \mathbb{A}^1$;

    3. $S^1\wedge \mathbb{G}_m$.
\end{proposition}

As a result, these objects in $Sh_{\bullet}(Sm/S)$ well-define an object in $\mathcal{H}(S)_\bullet$, which we call ``tate sphere" or ``motivic sphere".
    
As in classic algebraic topology, the ``sphere" induces the infinite loop functor, which gives rise to the stable motivic homotopy theory.