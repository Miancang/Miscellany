% !TEX root = ../main.tex

\chapter{Introduction}
\section{Foreword}

The concept of ``motive" was introduced by Grothendieck who noticed a series of similarities in Weil cohomology theories. In his idea, he thought there should be an universal cohomology theory which every cohomolgy theory taking values in a $\mathbb{Q}$-category should pass through it.

This idea soon became so important in the research about Weil conjecture that people belive once the ``motive" is discovered, then the problem can be done.

Thanks to the effort of many mathematicians, we succeeded in constructing ``pure motives" for smooth projective varieties as well as ``mixed motives" for a more general cases.

One important observation in algebraic topology is that when we talk about cohomology theories, what really matters is the derived category. It was Voevodksy who introduced the ``motivic homotopy category" and prove the universal property of it, which is what we are interested in this paper.

\section{The main content of this paper}
In short, motivic homotopy theory concerns how to construct a algebraic topology theory on algebraic geometry. To give the readers an intuition, we include a correponding diagram below.

\begin{table}[ht]
\centering
\caption{Comparison between classic algebraic topology world and motivic world}
\begin{tabular}{|>{\centering\arraybackslash}m{4cm}|>{\centering\arraybackslash}m{4cm}|}
\hline
\textbf{Classic World} & \textbf{Motivic World} \\
\hline
$\mathscr{S}$ (Spaces) & $\mathcal{H}(S)$ (Motivic homotopy category) \\
\hline
$I$& $\mathbb{A}^1$ \\
\hline
$I$-homotopy equivalence & $\mathbb{A}^1$-homotopy equivalence \\
\hline
$S^1$& Tate sphere \\
\hline
$\textbf{Sptr}$ (Spectra) & $\mathcal{SH}(S)$ ( Stable motivic homotopy category) \\
\hline
Cohomology & Motivic cohomology \\
\hline
\end{tabular}
\end{table}

In Chapter 2, we will firstly recommend some preliminaries that are important to understand the topic we'll discuss. It mainly contains the language of $\infty-$categories, stable $\infty-$ categories and the theory of six functors.

In Chapter 3, we will consturct the unstable motivic homotopy category $\mathcal{H}(S)$ and explain the reasonability in the constuction. 

In Chapter 4, we will construct the stable motivic homotopy category $\mathcal{SH}(S)$, which is a slightly differnt from the standard stabilization process. The existence of two circles in $\mathcal{H}(S)$ give rise to bigraded cohomology theories in motivic world. We will also see those important properties(universal properties, six functor formalism) and some spectra which have essential applications in other fields.

\section{The significance of this article}
Since motivic homotopy theory is a topic developed in mid 1990s and the language where it is based has changed with the development of $\infty-$categories, there has been a lack of a very detailed introduction of this topic in modern language. This is what this paper is in intended for.

