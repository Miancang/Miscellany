% !TEX root = ../main.tex

\begin{abstract}[zh]
\addcontentsline{toc}{chapter}{摘 \quad 要}
本文是建立在$(\infty,1)$-范畴语言中的对母体同伦理论的概述。

对于一个Noetherian概形$S$,存在一个不稳定的母体同伦范畴$\mathcal{H}(S)$,它可以视为预层范畴$Sm/S$的反射局部化。$\mathcal{H}(S)$ 中存在两种圆,这诱导了双分次的母体稳定同伦范畴$\mathcal{SH}_{s,t}(S)$和母体稳定同伦范畴$\mathcal{SH}(S)$。存在一个母体版本的Brown表示定理,它将(双分次)母体谱和(双分次)母体上同调理论联系起来。母体稳定同伦范畴$\mathcal{H}(S)$具有一个泛性质使每个上同调理论都通过它。此外,$\mathcal{SH}(S)$上存在一个6-函子理论。
\end{abstract}

{
\newfontface{\arial}{Arial}[Scale=0.94]
\ctexset{chapter/format+={\arial}}
\begin{abstract}[en]
\addcontentsline{toc}{chapter}{ABSTRACT}

This paper is an overview of motivic homotopy theory in the language of $(\infty,1)$-categories.

  For a Noethrian scheme $S$, there exists an unsabtle motivic homotopy category $\mathcal{H}(S)$  which can be considered as a reflective localization of the presheaf category $Sm/S$.  $\mathcal{H}(S)$ admits two types of circles, which induce bigraded motivic stable homotopy category $\mathcal{SH}_{s,t}(S)$ and motivic stable homotopy category $\mathcal{SH}(S)$. There is a motivic version Brown representibility theorem, which connects (bigraded) motivic spectra and (bigraded) motivic cohomology theory. The motivic stable homotopy category $\mathcal{SH}(S)$ has an universal property that every cohomology theory passes through it. Furthermore,  $\mathcal{SH}(S)$ admits a 6-functor formalism on it.

\end{abstract}
}