\chapter{Stable Motivic Homotopy Theory}

As we've shown in \ref{stable}, their is a standard process to get a stable $(\infty,1)$-category from a pointed $(\infty,1)$-category. The process happens in this chapter is a little different.

Instead of using the $(\infty,1)$-pushout of $0\gets X\to0$ (which concides with $X\wedge S^1$ in $\mathscr{S}_\bullet$), we use the original definition of suspension functor, which is just the smash product with ``circle".

However, we have two kinds of circles in motivic homotopy category, so correspondingly there are two types of suspension functors, which naturally lead to bispectrum as well as bicohomology.

The stabilization process gives rise to the stable motivic homotopy category $\mathcal{SH}(S)$. Robalo shows\cite{Robalo2015} the universal property of it, which show the reasonability of ``motivic" in its name.

\section{Stabilization}

This section is just a generalized version for biindexes of stabilzation process in higher algebra.

\begin{remark}
    Denote $S_t^1$ as the tate sphere, $S_s^1$ as the standard sphere.
\end{remark}

\begin{definition}
    The s-suspension functor $\Sigma_s:\mathcal{H}(S)_\bullet \to \mathcal{H}(S)_\bullet$ is defined as the functor: $-\wedge S_s^1$, while the t-suspension functor $\Sigma_t:\mathcal{H}(S)_\bullet \to \mathcal{H}(S)_\bullet$ is defined as the functor: $-\wedge S_t^1$.
\end{definition}

\begin{definition}
    The s-loop functor is defined as $\text{Maps}(S_s^1,-)$, while the t-loop functor is defined as $\text{Maps}(S_t^1,-)$.
\end{definition}

\begin{proposition}
    There are two adjunctions:

    $\Sigma_s: \mathcal{H}(S)_\bullet \rightleftarrows \mathcal{H}(S)_\bullet: \Omega_s$;

    $\Sigma_t: \mathcal{H}(S)_\bullet \rightleftarrows \mathcal{H}(S)_\bullet: \Omega_t$.
\end{proposition}


\begin{definition}\label{shs}
    The stable motivic homotopy category $\mathcal{SH}_s(S)$ (or simply denoted as $\mathcal{SH}(S)$) is defined as:

$    \operatorname{colim}[\mathcal{H}(S)\bullet \xrightarrow{\Sigma_s} \mathcal{H}(S)\bullet \xrightarrow{\Sigma_s} \cdots]\in \text{Pr}^L$

While the $\mathcal{SH}_t(S)$ can be defined as:

$    \operatorname{colim}[\mathcal{H}(S)\bullet \xrightarrow{\Sigma_t} \mathcal{H}(S)\bullet \xrightarrow{\Sigma_t} \cdots]\in \text{Pr}^L$

The object in motivic homotopy category is called ``motivic spectra".

\end{definition}

In duality, we can also define $    \operatorname{lim}[\mathcal{H}(S)\bullet \xleftarrow{\Omega_s} \mathcal{H}(S)\bullet \xleftarrow{\Omega_s} \cdots]\in \text{Pr}^R$, $    \operatorname{lim}[\mathcal{H}(S)\bullet \xleftarrow{\Omega_t} \mathcal{H}(S)\bullet \xleftarrow{\Omega_t} \cdots]\in \text{Pr}^R$. Since $\mathcal{H}(S)_\bullet$ is presentable, we have similar propositions like \ref{duality}.

\begin{proposition}

    The colimit $    \operatorname{colim}[\mathcal{H}(S)\bullet \xrightarrow{\Sigma_s} \mathcal{H}(S)\bullet \xrightarrow{\Sigma_s} \cdots]$ corresponds to the limit $\operatorname{lim}[\mathcal{H}(S)\bullet \xleftarrow{\Omega_s} \mathcal{H}(S)\bullet \xleftarrow{\Omega_s} \cdots]$ via $\text{Pr}^L\simeq (\text{Pr}^R)^{op}$;

    The colimit $    \operatorname{colim}[\mathcal{H}(S)\bullet \xrightarrow{\Sigma_t} \mathcal{H}(S)\bullet \xrightarrow{\Sigma_t} \cdots]$ corresponds to the limit $\operatorname{lim}[\mathcal{H}(S)\bullet \xleftarrow{\Omega_t} \mathcal{H}(S)\bullet \xleftarrow{\Omega_t} \cdots]$ via $\text{Pr}^L\simeq (\text{Pr}^R)^{op}$.
\end{proposition}

So we will not distingush the colimit and limit and call them all as $\mathcal{SH}_s(S)$ and $\mathcal{SH}(S)$.

\begin{remark}
    $\Sigma_{*}:\mathcal{H}(S)_\bullet\to \mathcal{SH}_{*}(S)$ and $\Omega_{*}:\mathcal{H}(S)_\bullet\to \mathcal{SH}_{*}(S)$ each natrually induce functor from $\mathcal{SH}_{*}(S)$ to $\mathcal{SH}_{*}(S)$ $(*\in \{s,t\})$. With a little abuse of notation, we still denote them as $\Sigma_{*} $ and $\Omega_{*}$.
\end{remark}


\begin{remark}
    
Since the stabilization precess we've mentioned in \ref{shs} has functorality, the process can be considered as a functor from $\mathcal{H}(S)_\bullet$ to $\mathcal{SH}(S)$, which we denoted as $\Sigma_*^\infty$, $*\in\{s,t\}$.

Duality, we denote the functor from $\mathcal{SH}(S)$  to $\mathcal{H}(S)_\bullet$ which maps a spectrum to its first term.
\end{remark}

\begin{remark}
    We will denote$\Sigma_*[k]$  as $\Sigma_*^k$, $\Omega_*[k]$ as $\Omega_*^k,*\in\{s,t\}$;

    We will denote $\Sigma_*^\infty\circ\Sigma_*[k]$  as $\Sigma_*^{\infty-k}$, $\Omega_*^\infty\circ\Omega_*[k]$  as $\Omega_*^{\infty-k},*\in\{s,t\}$.

    It's easy to verify the commutativity of these functors, which is why we use additivity to represent its index.
\end{remark}

\begin{proposition}
    There are two adjunctions:

$    \Sigma_*^{\infty-k}: \mathcal{H}(S)\bullet \rightleftarrows \mathcal{SH}_*(S): \Omega_*^{\infty-k}$ , $*\in \{s,t\}$;
\end{proposition}

As for bigraded cases, the situation become a little more complicated.

\begin{definition}
The category $\mathcal{SH}_{s,t}(S)$ can be defined as:
\[
\mathop{\mathrm{lim}}\limits_{\substack{\longrightarrow \\ \Sigma_s, \Sigma_t}} 
\left[
\begin{tikzcd}[
  row sep=1.5em, 
  column sep=1.5em,
  cells={nodes={minimum width=2em}} % 统一节点宽度
]
\cdots \arrow[r, "\Sigma_s"] & \cdots \arrow[r, "\Sigma_s"] & \cdots \arrow[r, "\Sigma_s"] & \cdots \\
\mathcal{H}(S)_\bullet \arrow[r, "\Sigma_s"] \arrow[u, "\Sigma_t"'] & 
\mathcal{H}(S)_\bullet \arrow[r, "\Sigma_s"] \arrow[u, "\Sigma_t"'] & 
\mathcal{H}(S)_\bullet \arrow[r, "\Sigma_s"] \arrow[u, "\Sigma_t"'] & 
\cdots \arrow[u, "\Sigma_t"'] \\
\mathcal{H}(S)_\bullet \arrow[r, "\Sigma_s"] \arrow[u, "\Sigma_t"'] & 
\mathcal{H}(S)_\bullet \arrow[r, "\Sigma_s"] \arrow[u, "\Sigma_t"'] & 
\mathcal{H}(S)_\bullet \arrow[r, "\Sigma_s"] \arrow[u, "\Sigma_t"'] & 
\cdots \arrow[u, "\Sigma_t"'] \\
\mathcal{H}(S)_\bullet \arrow[r, "\Sigma_s"] \arrow[u, "\Sigma_t"'] & 
\mathcal{H}(S)_\bullet \arrow[r, "\Sigma_s"] \arrow[u, "\Sigma_t"'] & 
\mathcal{H}(S)_\bullet \arrow[r, "\Sigma_s"] \arrow[u, "\Sigma_t"'] & 
\cdots \arrow[u, "\Sigma_t"']
\end{tikzcd}
\right]\in \text{Pr}^L

\]

\end{definition}

Similarily we can define $  \Sigma_{s,t}^{\infty,\infty}$ as the stabilization functor and $\Omega_{s,t}^{\infty,\infty}$ as taking the $(1,1)-$term of the spectrum. The definition of $\Sigma_{s,t}^{i,j}$ as well as $\Omega_{s,t}^{i,j}$ is similar.

\begin{proposition}
    $    \Sigma_{s,t}^{\infty-i,\infty-j}: \mathcal{H}(S)\bullet \rightleftarrows \mathcal{SH}_{s,t}(S): \Omega_{s,t}^{\infty-i,\infty-j}$.
\end{proposition}

\section{Motivic cohomology}

In classic algebraic topology, Brown representation theorem tell us that there exists a 1-1 corresponding between cohomology theories and spectra. Naumann and Spitzweck proved\cite{Naumann2009BrownRI} that the theorem still holds in motivic world, after giving the axioms which a bigraded cohomology theory should satisfy.

\begin{definition}
    We call a sequnce of functor $(F^{p,q})_{(p,q\in \mathbb{Z}_{\geq 0}^2)}:(Sm/S)^{op}\to \mathcal{E}ns$ is a motivic cohomology if:

    1. $F^{p+1,q}(S_s^1\wedge X)= F^{p,q}(X)$;

    2. $F^{p,q+1}(S_t^1\wedge X)= F^{p,q}(X)$;

    3. $F^{p,q}$ satisfies Nisnevich excision property and $\mathbb{A}^1-$invariant property. 
\end{definition}

\begin{theorem}
    A motivic cohomology is represented by a motivic spectrum $E$ by $F^{p,q}(X):=[X,\Omega_{s,t}^{\infty-(p-2q),\infty-q}E]=[\Sigma_{s,t}^{\infty}X,\Omega_{s,t}^{p-2q,q}E]$, and vice versa.
\end{theorem}

We will show some classic examples of motivic spectra together with their corresponding motivic cohomology.

\section{Motivic Spectra}

We firstly see the motivic Eilenberg-MacLane spectrum. The construction below come from \textcite{dundas2007motivic}.

\begin{definition}
    Let $X\in Sm/S$, define $L(X):(Sm/S)^{op}\to Ab$ which maps:

    $U\mapsto F(Irr(U,X))$,
    
    where $Irr(U,X)$ is the set of closed irreducible subscheme $Y$ of $U\times X$ such that $p_0:Y\to U$ is of finite type and $im(p_0)$ covers one of the connected compnents of $U$, $F$ is the free abelization functor.
\end{definition}

\begin{proposition}
    The functor $U\circ L$ is a Nisnevich sheaf which satisfies $\mathbb{A}^1-$equivalent category, where $U$ is the forgetful functor. Thus $U\circ L\in \mathcal{H}(S)$.
\end{proposition}

\begin{definition}
    The Eilenberg-MacLane spectrum $\mathbf{H}\mathbb{Z}^{p,q}:=  L\left((S_s^1)^{\wedge p} \wedge (S_t^1)^{\wedge q}\right)$.

\end{definition}

We don't forget its group structure here because we want to tansfer the cohomology theory.

Just as in classic algebraic topology, Eilenberg-MacLane spectrum represents the ``singular cohomology", that is, the Bloch's higher Chow group.

\begin{proposition}
    $[X,\mathbf{H}\mathbb{Z}^{p,q}]= CH^q(X,2q-p)$.
\end{proposition}

The next example we're cared about is the algebaric K-theory spectrum.

\begin{definition}
    Define $\text{BGL}_n:=\operatorname{colim} (...\to\text(Gr_n(\mathbb{A}^{n+m}))\to \text(Gr_n(\mathbb{A}^{n+m+1}))\to...)$;

    $\text{KGL}_n:= \mathbb{Z}\times \text{BGL}_n$.

    $\text{KGL}:=\operatorname{colim}[...\to\text{KGL}_{n-1}\to \text{KGL}_n\to \text{KGL}_{n+1}\to...]$
\end{definition}

\begin{proposition}
    There exists $\beta: \mathbb{P}^1\wedge \text{KGL}\to \text{KGL}$.
\end{proposition}

\begin{definition}
    We call the spectrum $(\text{KGL},\text{KGL},...)$ together with $\beta$ on each degree the algebraic K-theory spectrum.
\end{definition}

\begin{proposition}
    The algebraic K-theory spectrum represents algebraic K-theory.
\end{proposition}

\section{Universal Property}

In this section, we will show that the stable motivic homotopy category $\mathcal{SH}(S)$ is indeed ``motivic",i.e. has the universal property, which is proved by Robalo in 2012\cite{Robalo2012arxiv}.

\begin{definition}
    Let $X,Y \in \mathcal{SH}(S),X \otimes Y:=\lim _n n \lim _m\left(\Sigma^{\infty}\left(X_n \wedge Y_m\right)[-(n+m)]\right)$
\end{definition}

\begin{definition}
    Let $X,Y \in \mathcal{H}(S), map(X,Y):=([X,Y[n]],[[X,Y[n]]\to [X,\Omega Y[n+1]]\to \Omega[X,Y[n+1]]])$.
\end{definition}

\begin{proposition}
    The $\otimes$ functor and the $map$ functor define a symmetric monoidal closed structure on $\mathcal{SH}(S)$.
\end{proposition}

Since $\mathcal{H}(S)$ is presentable, we have:

\begin{proposition}
    $\mathcal{SH}(S)$ is pointed and presentable.
\end{proposition}

\begin{proposition}
    The composition of functor $L_{\mathbb{A}^1}\circ()_+\circ h_C$: $Sm/S \to \mathcal{SH}(S)$ is a symmetric monoidal, colimit-preserving functor.
\end{proposition}

\begin{theorem}\cite{Robalo2012arxiv}
    Let $D^{\otimes}$ be a pointed presentable symmetric monoidal $(\infty,1)-$category, then the composition map:

    $Fun^{\otimes,L}(\mathcal{SH}(S)^\otimes,D^\otimes)\to Fun^{\otimes,L}(Sm/S^\times,D^\otimes)$ 

    is fully faithful and its image satisfies:

    1.Nisnevich excision property;

    2.$\mathbb{A}^1$-invariant; 

    3.send Tate sphere to an invertible object.

    
\end{theorem}

This theorem completes the original motivation for our topic, i.e. find a universal cohomology theory such that every cohomology theory passes through it.

\section{Six functor fomalism}

In 2007, Ayoub \cite{Ayoub2007} found that the stable motivic homotopy category $\mathcal{SH}(S)$ admits a six functor theory, which can be seen as an enhance of cohomology theory.

\begin{theorem}\cite{Ayoub2007}
    There is a six functor theory on $(Sm/S,\mathcal{SH}(S))$, where $E$ is the set of separated morphisms of finite type.
\end{theorem}

