\documentclass[12pt, reqno]{amsart}
\usepackage{amsmath, amsthm, amscd, amsfonts, amssymb, graphicx, xcolor, tikz-cd}
\usepackage[bookmarksnumbered, colorlinks, plainpages]{hyperref}

\textheight 22.5truecm \textwidth 14.5truecm
\setlength{\oddsidemargin}{0.35in}\setlength{\evensidemargin}{0.35in}

\setlength{\topmargin}{-.5cm}

\newtheorem{theorem}{Theorem}[section]
\newtheorem{lemma}[theorem]{Lemma}
\newtheorem{proposition}[theorem]{Proposition}
\newtheorem{corollary}[theorem]{Corollary}
\theoremstyle{definition}
\newtheorem{definition}[theorem]{Definition}
\newtheorem{example}[theorem]{Example}
\newtheorem{exercise}[theorem]{Exercise}
\newtheorem{conclusion}[theorem]{Conclusion}
\newtheorem{conjecture}[theorem]{Conjecture}
\newtheorem{criterion}[theorem]{Criterion}
\newtheorem{summary}[theorem]{Summary}
\newtheorem{axiom}[theorem]{Axiom}
\newtheorem{problem}[theorem]{Problem}
\theoremstyle{remark}
\newtheorem{remark}[theorem]{Remark}
\numberwithin{equation}{section}

\begin{document}
\setcounter{page}{1}

\color{darkgray}{
\noindent 


\centerline{}

\centerline{}


\title[Directed Study on Spectral Sequences in Algebraic Topology]{Directed Study on Spectral Sequences in Algebraic Topology}

\author{Xingzhi Huang}


 \maketitle

\section{Goals}

This directed study is mainly based on the book of Allen Hatcher \cite{HatSS} focusing on the Serre spectral sequence and Adams spectral sequence, but also includes some other materials to clerify some concepts and provide some new tools to compute the Adams spectral sequence. The leading question of the study is to compute the homotopy groups of spheres.
\section{Outline}

\subsection*{\textbf{§1: Serre Spectral Sequence}}

Given a fibration $F \to E \to B$ with $B$ simply connected, there is a spectral sequence with
\[  E^2_{p,q} \cong H_p(B; H_q(F; G))\Rightarrow H_{p+q}(E; G) \]
for any abelian group $G$. This spectral sequence is called the Serre spectral sequence.

The construction of the Serre spectral sequence has naturality from the fiber sequence to the spectral sequence.

We will also see several generalizations of the Serre spectral sequence, including the case when $B$ is not simply connected where we have to repleace the homology with local systems, and the case of generalized homology where we have to consider Atiyah-Hirzebruch spectral sequence.

For cohomology, expect of 
\[  E_2^{p,q} \cong H^p(B; H^q(F; G))\Rightarrow H^{p+q}(E; G) \]

We will also have the multiplicative structure with it.

Apply the Serre spectral sequence to the path fibration $K(\mathbb{Z}, n-1) \to P \to K(\mathbb{Z}, n)$ with coefficient $\mathbb{Q}$, we can compute the cohomology of Eilenberg-MacLane spaces: 

$H^*(K(\mathbb{Z}, n); \mathbb{Q}) = \begin{cases}\Lambda_{\mathbb{Q}}[x] & n \text{ odd}\\
\mathbb{Q}[x] & n \text{ even}\end{cases}$

A direct corollary of this result lies in the homotopy groups of spheres: $\pi_i(S^n)$ is finite for $i>n$ except for $i=2n-1$ when $n$ is even.

Reference: \cite{HatSS}, \cite{May11}

\subsection*{\textbf{§2: Hopf Algebra, H-Space and H-Group}}

Given a braided monoidal category $\mathcal{C}$, we will see $\mathrm{Alg}(\mathcal{C})$ still has monoidal structure, thus we can define the coalgebra of it, which we call as the bialgebra of $\mathcal{C}$. Furthermore, if we have the antipodal structure, we can call it the Hopf algebra.

Given a space $X$ with coefficient $R$, the ring structure of cohomology naturally gives the algebra structure of $H^*(X; R)$. If we give $X$ a kind of "weak" algebra structure on $\mathrm{Top}$, i.e. H-space structure, then we can get the coalgebra structure of $H^*(X; R)$; If we further give $X$ a kind of inverse structure, i.e. H-group sturcture, then we can get the antipodal structure of $H^*(X; R)$. Thus we have the Hopf algebra structure of $H^*(X; R)$.

Reference: \cite{OAMM}, \cite{Hat02}

\subsection*{\textbf{§3: Localization of Spaces and Rational Homotopy Groups of Spheres}}

Given a set of primes $\mathcal{P}$ and a finitely generated abelian group $G$, we can construct the $\mathcal{P}$-localization of $G$, $G_{\mathcal{P}}$, to kill the torsion part of $G$ whose order is not divided by any prime in $\mathcal{P}$. For a space $X$, the localization of its homotopy groups can be lifted to the space level, i.e. for a space $X$, there is a space $X_{\mathcal{P}}$ such that $\pi_i(X_{\mathcal{P}}) \cong \pi_i(X)_{\mathcal{P}}$. 

One of the applications of localization lies in the case where $\mathcal{P}$ is empty(which we call as rational localization) and $X$ is a $H$-group with finitely generated homotopy groups. In this case, we have $H^*(X; \mathbb{Q}) = \mathrm{Sym}(\pi_*(X) \otimes \mathbb{Q})$.

Take $X=S^n$, we have $\pi_i(S^n) \otimes \mathbb{Q} = \begin{cases}\mathbb{Q} & i=n \text{ or } i=2n-1 \text{ when } n \text{ is even}\\
0 & \text{ otherwise}\end{cases}$

This encodes the the non-torsion part of homotopy groups of spheres.

Reference: \cite{May11}, \cite{HatSS}

\subsection*{\textbf{§4: Cohomology Operation, Steenrod Square and Steenrod Algebra}}

A cohomology operation of type $(\pi, n, G, m)$ is a family of morphisms \[ \theta_X: H^n(X; \pi) \to H^m(X; G) \] which is natural with respect to the base space. We further call a cohomology operation stable if it commutes with the suspension. The Steenrod algebra is the algebra of stable cohomology operations with coefficient $\mathbb{Z}_2$.

We could establish a bijection between the set of cohomology operations of type $(\pi, n, G, m)$ and $H^m(K(\pi, n); G)$. Thus when $\pi=G=\mathbb{Z}_2$, we will see the only nontrivial cohomology operations are the Steenrod squares $Sq^i: H^n(X; \mathbb{Z}_2) \to H^{n+i}(X; \mathbb{Z}_2)$. The Steenrod algebra $\mathcal{A}_2$ is isomorphic to the algebra generated by all Steenrod squares with the Adem relations: 

\[ Sq^i Sq^j = \sum_{k=0}^{\lfloor i/2 \rfloor} \binom{j-k-1}{i-2k} Sq^{i+j-k} Sq^k, \quad i<2j \]

We will see that the Steenrod algebra $\mathcal{A}_2$ plays an important role in calculating the Eilenberg-MacLane spaces with coefficient $\mathbb{Z}_2$, which gives rise to some important results of the 2-torsion part of homotopy groups of spheres. We will finally see that $\pi_{n+2}(S^n) \cong \mathbb{Z}_2$ for $n \geq 2$.

Reference: \cite{Hat02}, \cite{MT08}

\subsection*{\textbf{§5: Adams Spectral Sequence and the Stable Homotopy Groups of Spheres}}

Given a spectrum $E$ under mild assumptions, there is a spectral sequence with
\[ E_2^{s,t} \cong \mathrm{Ext}^{s,t}_{\mathcal{A}_p}(H^*(E; \mathbb{Z}_p), \mathbb{Z}_p) \Rightarrow \pi_{t-s}(E) \otimes \mathbb{Z}_p \]
for any prime $p$. This spectral sequence is called the Adams spectral sequence. Take $E$ as the sphere spectrum, we can use this spectral sequence to compute the stable homotopy groups of spheres.

To compute the $E_2$-page, a useful tool is the May spectral sequence, which is a spectral sequence with
\[ E_1^{s,t} \cong \mathrm{Ext}^{s,t}_{\mathrm{Gr}(\mathcal{A}_p)}(\mathbb{Z}_p, \mathbb{Z}_p) \Rightarrow \mathrm{Ext}^{s,t}_{\mathcal{A}_p}(\mathbb{Z}_p, \mathbb{Z}_p) \]
where $\mathrm{Gr}(\mathcal{A}_p)$ is the associated graded algebra of $\mathcal{A}_p$ with respect to the May filtration.  

Reference: \cite{HatSS}, \cite{Koc96}

\subsection*{\textbf{§6: Chromatic Homotopy Theory and Chromatic Spectral Sequence}}

\textcolor{red}{[Attention: I have no idea of this section yet, this part is from wiki]}

A chromatic spectral sequence is a spectral sequence of a filtered stable homotopy type for the case of a filtering given by a chromatic tower.This is a type of spectral sequence useful for computing the $E_1$ term of the Adams-Novikov spectral sequence.

Reference: \cite{Rog23}


\bibliographystyle{amsplain}
\begin{thebibliography}{99}
\bibitem[HatSS]{HatSS}
Hatcher, Allen.
\emph{Spectral Sequences}.
preprint, 2004.
\url{https://pi.math.cornell.edu/~hatcher/AT/ATch5.pdf}

\bibitem[May11]{May11}
May, J. Peter, and Ponto, Kate.
\emph{More Concise Algebraic Topology: Localization, Completion, and Model Categories}.
University of Chicago Press, 2011.
\url{https://webhomes.maths.ed.ac.uk/~v1ranick/papers/mayponto.pdf}

\bibitem[OAMM]{OAMM}
Kriz, Igor and May, J. Peter,
\emph{Operads, algebras, modules and motives},
Ast\'{e}risque, vol. 233,
Soci\'{e}t\'{e} Math\'{e}matique de France, Paris, 1995.
\url{https://www.numdam.org/issue/AST_1995__233__1_0.pdf}

\bibitem[Hat02]{Hat02}
Hatcher, Allen,
\emph{Algebraic Topology},
Cambridge University Press, Cambridge, 2002.
\url{https://pi.math.cornell.edu/~hatcher/AT/AT+.pdf}

\bibitem[MT08]{MT08}
Mosher, Robert E. and Tangora, Martin C.,
\emph{Cohomology operations and applications in homotopy theory},
Courier Corporation, Mineola, NY, 2008.
\url{https://webhomes.maths.ed.ac.uk/~v1ranick/papers/moshtang.pdf}

\bibitem[Koc96]{Koc96}
Kochman, Stanley O.,
\emph{Bordism, stable homotopy and Adams spectral sequences},
Fields Institute Monographs, vol. 7,
American Mathematical Society, Providence, RI, 1996.

\bibitem[Rog23]{Rog23}
Rognes, John,
\emph{Chromatic homotopy theory},
Lecture notes, Algebraic Topology III (MAT4580/MAT9580),
University of Oslo, Oslo, Norway, Spring 2023.
\url{https://www.uio.no/studier/emner/matnat/math/MAT9580/v23/documents/chromatic.pdf}

\end{thebibliography}


\end{document}