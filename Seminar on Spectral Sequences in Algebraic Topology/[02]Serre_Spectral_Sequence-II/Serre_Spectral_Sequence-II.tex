\documentclass[12pt, reqno]{amsart}
\usepackage{amsmath, amsthm, amscd, amsfonts, amssymb, graphicx, xcolor, tikz-cd}
\usepackage[bookmarksnumbered, colorlinks, plainpages]{hyperref}

\textheight 22.5truecm \textwidth 14.5truecm
\setlength{\oddsidemargin}{0.35in}\setlength{\evensidemargin}{0.35in}

\setlength{\topmargin}{-.5cm}

\newtheorem{theorem}{Theorem}[section]
\newtheorem{lemma}[theorem]{Lemma}
\newtheorem{proposition}[theorem]{Proposition}
\newtheorem{corollary}[theorem]{Corollary}
\theoremstyle{definition}
\newtheorem{definition}[theorem]{Definition}
\newtheorem{example}[theorem]{Example}
\newtheorem{exercise}[theorem]{Exercise}
\newtheorem{conclusion}[theorem]{Conclusion}
\newtheorem{conjecture}[theorem]{Conjecture}
\newtheorem{criterion}[theorem]{Criterion}
\newtheorem{summary}[theorem]{Summary}
\newtheorem{axiom}[theorem]{Axiom}
\newtheorem{problem}[theorem]{Problem}
\theoremstyle{remark}
\newtheorem{remark}[theorem]{Remark}
\numberwithin{equation}{section}

\begin{document}
\setcounter{page}{1}

\color{darkgray}{
\noindent 


\centerline{}

\centerline{}


\title[Serre Spectral Sequence-II]{Serre Spectral Sequence-II}

\author{Xingzhi Huang}

 \maketitle

\section{Construction of Serre Spectral Sequence-Proof}

As we've shown, the first page of the Serre spectral sequence is given by
\[E^1_{p, q} \cong H_{p+q}(p^{-1}(B^p), p^{-1}(B^{p-1});R).\]
where the differential $d^1$ is the $j\circ k$, which are induced from the long exact sequence of the triple

\[(p^{-1}(B^p), p^{-1}(B^{p-1}), p^{-1}(B^{p-2})).\]

Another way to see the first page is to extend it to the "zero" page and view $E^1$ as the homology of $E^0$ with respect to $d^0$. Clearly, we can define $E^0_{p, q} := C_{p+q}(p^{-1}(B^p));R)$

\bibliographystyle{amsplain}
\begin{thebibliography}{99}
\bibitem{Li2024}
Wenwei Li.
\emph{Methods of Algebra, Vol. 2: Linear Algebra}.
Higher Education Press, 2024. (in Chinese)

\bibitem{McCleary2001}
McCleary, John.
\emph{A User's Guide to Spectral Sequences}, 2nd ed.
Cambridge University Press, 2001.
\end{thebibliography}


\end{document}