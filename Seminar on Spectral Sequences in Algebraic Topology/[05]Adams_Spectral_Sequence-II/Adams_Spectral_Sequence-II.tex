\documentclass[12pt, reqno]{amsart}
\usepackage{amsmath, amsthm, amscd, amsfonts, amssymb, graphicx, xcolor, tikz-cd}
\usepackage[bookmarksnumbered, colorlinks, plainpages]{hyperref}

\textheight 22.5truecm \textwidth 14.5truecm
\setlength{\oddsidemargin}{0.35in}\setlength{\evensidemargin}{0.35in}

\setlength{\topmargin}{-.5cm}

\newtheorem{theorem}{Theorem}[section]
\newtheorem{lemma}[theorem]{Lemma}
\newtheorem{proposition}[theorem]{Proposition}
\newtheorem{corollary}[theorem]{Corollary}
\theoremstyle{definition}
\newtheorem{definition}[theorem]{Definition}
\newtheorem{example}[theorem]{Example}
\newtheorem{exercise}[theorem]{Exercise}
\newtheorem{conclusion}[theorem]{Conclusion}
\newtheorem{conjecture}[theorem]{Conjecture}
\newtheorem{criterion}[theorem]{Criterion}
\newtheorem{summary}[theorem]{Summary}
\newtheorem{axiom}[theorem]{Axiom}
\newtheorem{problem}[theorem]{Problem}
\theoremstyle{remark}
\newtheorem{remark}[theorem]{Remark}
\numberwithin{equation}{section}

\begin{document}
\setcounter{page}{1}

\color{darkgray}{
\noindent 


\centerline{}

\centerline{}


\title[Adams Spectral Sequence-II]{Adams Spectral Sequence-II}

\author{Xingzhi Huang}

 \maketitle

\section{Review}

We first review some basic preliminaries on the lecture of Adams spectral sequence-I.

\subsection{Spectrum}

\begin{definition}
We can define a \textbf{spectrum} $E$ as the following two equivalent ways:

\begin{itemize}
    \item A sequence of pointed spaces $\{E_n\}_{n\in \mathbb{Z}}$ together with morphisms $\sigma_n: \Sigma E_n \to E_{n+1}$.
    \item A sequence of pointed spaces $\{E_n\}_{n\in \mathbb{Z}}$ together with isomorphisms $\tilde{\sigma}_n: E_n \xrightarrow{\sim} \Omega E_{n+1}$.
\end{itemize}
\end{definition}

\begin{definition}[Generalized Cohomology \& Homology Theory]
    We call a functor $h^*: h \mathcal{S}_*^{op} \to \text{Ab}^\mathbb{Z}$ a \textbf{generalized cohomology theory} if it satisfies:

    1. Suspension Axiom: For any space $X$, there is a natural isomorphism $h^*(X) \cong h^{*+1}(\Sigma X)$.

    2. Exactness Axiom: For any cofiber sequence $X \to Y \to Z$, the induced sequence
    \[\cdots \to h^*(Z) \to h^*(Y) \to h^*(X) \to \cdots\]
    is exact.

    3. Wedge Axiom: For any family of spaces $\{X_\alpha\}_{\alpha \in A}$, the natural map
    \[h^*\left(\bigvee_{\alpha \in A} X_\alpha\right) \to \prod_{\alpha \in A} h^*(X_\alpha)\]
    is an isomorphism.

    We call a functor $h_*: h \mathcal{S}_* \to \text{Ab}^\mathbb{Z}$ a \textbf{generalized homology theory} if it satisfies:

    1. Suspension Axiom: For any space $X$, there is a natural isomorphism $h_*(X) \cong h_{*+1}(\Sigma X)$.

    2. Exactness Axiom: For any cofiber sequence $X \to Y \to Z$, the induced sequence
    \[\cdots \to h_*(X) \to h_*(Y) \to h_*(Z) \to \cdots\]
    is exact.

    3. Wedge Axiom: For any family of spaces $\{X_\alpha\}_{\alpha \in A}$, the natural map
    \[h_*\left(\bigvee_{\alpha \in A} X_\alpha\right) \to \bigoplus_{\alpha \in A} h_*(X_\alpha)\]
    is an isomorphism.

\end{definition}

\begin{theorem}[Brown Representability Theorem]
    Any generalized cohomology theory $h^*$ is representable, i.e. there exists a spectrum $E$ such that for any pointed space $X$, there is a natural isomorphism
    \[h^n(X) \cong [X, E_n].\]

    Any generalized homology theory $h_*$ is representable, i.e. there exists a spectrum $E$ such that for any pointed space $X$, there is a natural isomorphism
    \[h_n(X) \cong \varinjlim_k [S^k, X \wedge E_{n+k}].\]
\end{theorem}

\begin{remark}
    A more natural way to present looks like:

    \[ h^n(X) \cong \pi_{-n}\text{Maps}(\Sigma^\infty X, E) \]

    \[ h_n(X) \cong \pi_n \text{Maps}(\Sigma^\infty X \wedge E) \]
\end{remark}

\begin{theorem}[Symmetric Monoidal Structure on Spectra]

\end{theorem}

\begin{definition}[Ring Spectrum]

\end{definition}
\subsection{Eilenberg-MacLane Spectrum}

\begin{definition}[Delooping]
\end{definition}

\begin{definition}[Eilenberg-MacLane Space]
\end{definition}

\begin{definition}[Eilenberg-MacLane Spectrum]
\end{definition}

\begin{theorem}[Representability of Singular Cohomology]
\end{theorem}

\begin{theorem}[Ring Structure on Eilenberg-MacLane Spectrum]
\end{theorem}

\subsection{Cohomology Operations and Stable Cohomology Operations}

\begin{definition}[Cohomology Operation]
    A \textbf{cohomology operation} of type $(n, G) \to (m, G')$ is a natural transformation between two cohomology functors, i.e. a collection of maps
    \[\{\theta_X: H^n(X; G) \to H^m(X; G')\}_{X \in \text{Top}_*}\]
    such that for any pointed map $f: X \to Y$, the following diagram commutes:
    \[\begin{tikzcd}
    H^n(Y; G) \arrow{r}{\theta_Y} \arrow{d}{f^*} & H^m(Y; G') \arrow{d}{f^*} \\
    H^n(X; G) \arrow{r}{\theta_X} & H^m(X; G')
    \end{tikzcd}\]
\end{definition}

Notice that by the representability of singular cohomology, $H^n(-; G) $ as a functor is represented by $ [-, K(G,n)]$, thus a cohomology operation $\theta$ of type $(n, G) \to (m, G')$ corresponds to a natural transformation of functors \[[-, K(G,n)] \to [-, K(G',m)]\], thus by Yoneda lemma(in the homotopy category of pointed spaces), it corresponds to $H^m(K(G,n); G')$.

\begin{definition}[Stable Cohomology Operation]
    A \textbf{stable cohomology operation} of degree $k$ from $G$ to $G'$ is a collection of cohomology operations
    \[\{\theta^n: H^n(-; G) \to H^{n+k}(-; G')\}_{n \in \mathbb{Z}}\]
    such that for any pointed space $X$, the following diagram commutes:
    \[\begin{tikzcd}
    H^n(X; G) \arrow{r}{\theta^n_X} \arrow{d}{\Sigma} & H^{n+k}(X; G') \arrow{d}{\Sigma} \\
    H^{n+1}(\Sigma X; G) \arrow{r}{\theta^{n+1}_{\Sigma X}} & H^{n+k+1}(\Sigma X; G')
    \end{tikzcd}\]
\end{definition}

\section{Hopf Algebra}

\begin{definition}[Algebra and Coalgebra over a Monoidal Category]
\end{definition}

\begin{definition}[Bialgebra]
\end{definition}

\begin{example}[k-module]
\end{example}

\begin{definition}[Hopf Algebra]
\end{definition}
\section{Steenrod Algebra}

\begin{definition}[Steenrod Squares]
\end{definition}

\section{E-Adams Spectral Sequence}

\subsection{Classic Adams Spectral Sequence}

\bibliographystyle{amsplain}
\begin{thebibliography}{99}
\end{thebibliography}


\end{document}