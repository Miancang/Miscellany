\documentclass[12pt, reqno]{amsart}
\usepackage{amsmath, amsthm, amscd, amsfonts, amssymb, graphicx, xcolor, tikz-cd}
\usepackage[bookmarksnumbered, colorlinks, plainpages]{hyperref}

\textheight 22.5truecm \textwidth 14.5truecm
\setlength{\oddsidemargin}{0.35in}\setlength{\evensidemargin}{0.35in}

\setlength{\topmargin}{-.5cm}

\newtheorem{theorem}{Theorem}[section]
\newtheorem{lemma}[theorem]{Lemma}
\newtheorem{proposition}[theorem]{Proposition}
\newtheorem{corollary}[theorem]{Corollary}
\theoremstyle{definition}
\newtheorem{definition}[theorem]{Definition}
\newtheorem{example}[theorem]{Example}
\newtheorem{exercise}[theorem]{Exercise}
\newtheorem{conclusion}[theorem]{Conclusion}
\newtheorem{conjecture}[theorem]{Conjecture}
\newtheorem{criterion}[theorem]{Criterion}
\newtheorem{summary}[theorem]{Summary}
\newtheorem{axiom}[theorem]{Axiom}
\newtheorem{problem}[theorem]{Problem}
\theoremstyle{remark}
\newtheorem{remark}[theorem]{Remark}
\numberwithin{equation}{section}

\begin{document}
\setcounter{page}{1}

\color{darkgray}{
\noindent 


\centerline{}

\centerline{}


\title[Adams Spectral Sequence-II]{Adams Spectral Sequence-II}

\author{Xingzhi Huang}

 \maketitle

\section{Review}

We first review some basic preliminaries on the lecture of Adams spectral sequence-I.

\subsection{Spectrum}

\begin{definition}
We can define a \textbf{spectrum} $E$ as the following two equivalent ways:

\begin{itemize}
    \item A sequence of pointed spaces $\{E_n\}_{n\in \mathbb{Z}}$ together with morphisms $\sigma_n: \Sigma E_n \to E_{n+1}$.
    \item A sequence of pointed spaces $\{E_n\}_{n\in \mathbb{Z}}$ together with isomorphisms $\tilde{\sigma}_n: E_n \xrightarrow{\sim} \Omega E_{n+1}$.
\end{itemize}
\end{definition}

\begin{theorem}[Brown Representability Theorem]
    For any reduced     
\end{theorem}

\subsection{Eilenberg-MacLane Spectrum}

\subsection{Cohomology Operations and Stable Cohomology Operations}

\section{Hopf Algebra}

\section{Steenrod Algebra}

\section{E-Adams Spectral Sequence}

\subsection{Classic Adams Spectral Sequence}

\bibliographystyle{amsplain}
\begin{thebibliography}{99}
\end{thebibliography}


\end{document}